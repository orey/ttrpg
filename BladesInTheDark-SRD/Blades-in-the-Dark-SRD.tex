\documentclass[11pt,oneside]{book}

% Include all the required packages
\usepackage[utf8]{inputenc}
\usepackage[T1]{fontenc}
\usepackage{geometry}
\geometry{
 	a4paper,
 	total={170mm,257mm},
 	left=20mm,
 	top=20mm,
}
\usepackage{hyperref}
\hypersetup{
	pdftitle={Blades in the Dark SRD},
    pdfsubject={Blades in the Dark,SRD},
	pdfkeywords={ttrpg, blades},
	pdfsubject={ttrpg},
	pdflang={English},
	bookmarksopen=true,
	bookmarksopenlevel=0,
	bookmarksdepth=1,
	bookmarksnumbered=true,
	colorlinks=true,
	linkcolor={blue},
	urlcolor={blue},
}

\title{Blades in the Dark}
\date{December 28 2021}
\author{This work is based on \textit{Blades in the Dark} (\href{http://www.bladesinthedark.com}{bladesinthedark.com}), \\
product of \href{http://www.onesevendesign.com}{One Seven Design}, developed and authored by John Harper, \\
and licensed for our use under the \href{http://creativecommons.org/licenses/by/3.0/}{Creative Commons Att. 3.0 UL}. \\
Forked from \href{https://github.com/amazingrando/blades-in-the-dark-srd-content}{amazingrando/blades-in-the-dark-srd-content}\\
by \href{https://github.com/orey/jdr}{orey/jdr}}

% \gameterm formats special game-specific terminology
\newcommand{\gameterm}[1]{\textbf{#1}}


% suppress subsections and lower from the ToC
\setcounter{tocdepth}{1}

\begin{document}

\maketitle

\tableofcontents

\part{The Basics}

\section{The Game}

Blades in the Dark is a game about a group of daring characters building an enterprising crew. We play to find out if the fledgling crew can thrive amidst the teeming threats that surround it.

\section{The Players}

Each player creates a character and works with the other players to create the crew to which their characters belong. Each player strives to bring their character to life as an interesting, daring character who reaches boldly beyond their current safety and means.

The players work together with the Game Master to establish the tone and style of the game by making judgment calls about the mechanics, dice, and consequences of actions. The players take responsibility as co-authors of the game with the GM.

\section{The Characters}

The characters attempt to develop their crew by performing scores and contending with threats from their enemies.

\section{The Crew}

In addition to creating characters, you’ll also create the crew by choosing which type of criminal enterprise you’re interested in exploring.

\section{The Game Master}

The GM establishes the dynamic world around the characters. The GM plays all the non-player characters in the world by giving each one a concrete desire and preferred method of action.

The GM helps organize the conversation of the game so it’s pointed toward the interesting elements of play. The GM isn’t in charge of the story and doesn’t have to plan events ahead of time. They present interesting opportunities to the players, then follow the chain of action and consequences wherever they lead.

\section{Playing A Session}

A session of Blades in the Dark is like an episode of a TV show. There are one or two main events, plus maybe some side-story elements, which all fit into an ongoing series. A session of play can last anywhere from two to six hours, depending on the preferences of the group.

During a session, the crew of scoundrels works together to choose a score to accomplish, then they make a few dice rolls to jump into the action of the score in progress. The PCs take actions, suffer consequences, and finish the operation (succeed or fail). Then the crew has downtime, during which they recover, pursue side-projects, and indulge their vices. After downtime, the players once again look for a new opportunity or create their own goals and pursuits, and we play to find out what happens next.

\chapter{The Core System}

\section{Judgment Calls}

When you play, you’ll make several key judgment calls. Everyone contributes, but either the players or the GM gets final say for each:

\begin{itemize}
	\item Which actions are reasonable as a solution to a problem? \emph{Can this person be swayed? Must we get out the tools and tinker with this old rusty lock, or could it also be quietly finessed?} The players have final say.
	\item How dangerous and how effective is a given action in this circumstance? \emph{How risky is this? Can this person be swayed very little or a whole lot?} The GM has final say.
	\item Which consequences are inflicted to manifest the dangers in a given circumstance? \emph{Does this fall from the roof break your leg? Do the constables merely become suspicious or do they already have you trapped?} The GM has final say.
	\item Does this situation call for a dice roll, and which one? \emph{Is your character in position to make an action roll or must they first make a resistance roll to gain initiative?} The GM has final say.
	\item Which events in the story match the experience triggers for character and crew advancement? \emph{Did you express your character’s beliefs, drives, heritage, or background? You tell us.} The players have final say.
\end{itemize}

\section{Rolling the Dice}

Blades in the Dark uses six-sided dice. You roll several at once and read the \textbf{single highest result}.

\begin{itemize}
	\item If the highest die is a \textbf{6}, it’s a \textbf{full success}---things go well. If you roll more than one \textbf{6}, it’s a \textbf{critical success}---you gain some additional advantage.
	\item If the highest die is a \textbf{4 or 5}, that’s a \textbf{partial success}---you do what you were trying to do, but there are consequences: trouble, harm, reduced effect, etc.
	\item If the highest die is \textbf{1-3}, it’s a \textbf{bad outcome}. Things go poorly. You probably don’t achieve your goal and you suffer complications, too.
\end{itemize}

\emph{If you ever need to roll but you have zero (or negative) dice, roll two dice and take the single lowest result. You can’t roll a \textbf{critical} when you have zero dice.}

All the dice systems in the game are expressions of this basic format. When you’re first learning the game, you can always “collapse” back down to a simple roll to judge how things go. Look up the exact rule later when you have time.

To create a dice pool for a roll, you’ll use a \textbf{trait} (like your \gameterm{Finesse}  or your \gameterm{Prowess}  or your crew’s Tier) and take dice equal to its \textbf{rating}. You’ll usually end up with one to four dice. Even one die is pretty good in this game---a 50\% chance of success. The most common traits you’ll use are the \textbf{action ratings} of the player characters. A player might roll dice for their \gameterm{Skirmish}  action rating when they fight an enemy, for example.

There are four types of rolls that you’ll use most often in the game:

\begin{itemize}
	\item \gameterm{Action roll. } When a PC attempts an action that’s dangerous or troublesome, you make an action roll to find out how it goes. Action rolls and their effects and consequences drive most of the game.
	\item \gameterm{Downtime roll. } When the PCs are at their leisure after a job, they can perform downtime activities in relative safety. You make downtime rolls to see how much they get done.
	\item \gameterm{Fortune roll. } The GM can make a fortune roll to disclaim decision making and leave something up to chance. \emph{How loyal is an NPC? How much does the plague spread? How much evidence is burned before the constables kick in the door?}
	\item \gameterm{Resistance roll. } A player can make a resistance roll when their character suffers a consequence they don’t like. The roll tells us how much stress their character suffers to reduce the severity of a consequence. \emph{When you resist that “Broken Leg” harm, you take some stress and now it’s only a “Sprained Ankle” instead.}
\end{itemize}

\section{The Game Structure}

Blades in the Dark has a structure to play, with four parts. By default, the game is in \textbf{free play}---characters talk to each other, they go places, they do things, they make rolls as needed.

When the group is ready, they choose a \emph{target} for their next operation, then choose a type of \emph{plan} to employ. This triggers the \emph{engagement roll} (which establishes the situation as the operation starts) and then the game shifts into the \textbf{score} phase.

During the score, the PCs engage the target---they make rolls, overcome obstacles, call for flashbacks, and complete the operation (successfully or not). When the score is finished, the game shifts into the \textbf{downtime} phase.

During the downtime phase, the GM engages the systems for \emph{payoff}, \emph{heat}, and \emph{entanglements}, to determine all the fallout from the score. Then the PCs each get their \emph{downtime activities}, such as indulging their vice to remove stress or working on a long-term project. When all the downtime activities are complete, the game returns to \textbf{free play} and the cycle starts over again.

The phases are a conceptual model to help you organize the game. They’re not meant to be rigid structures that restrict your options (this is why they’re presented as amorphous blobs of ink without hard edges). Think of the phases as a menu of options to fit whatever it is you’re trying to accomplish in play. Each phase suits a different goal.

\chapter{Actions \& Attributes}

\section{Action Ratings}

There are 12 \textbf{actions} in the game that the player characters use to overcome obstacles.

\begin{itemize}
	\item Attune
	\item Command
	\item Consort
	\item Finesse
	\item Hunt
	\item Prowl
	\item Skirmish
	\item Study
	\item Survey
	\item Sway
	\item Tinker
	\item Wreck
\end{itemize}

Each action has a rating (from zero to 4) that tells you how many dice to roll when you perform that action. Action ratings don’t just represent skill or training---you’re free to describe \emph{how} your character performs that action based on the type of person they are. Maybe your character is good at \gameterm{Command}  because they have a scary stillness to them, while another character barks orders and intimidates people with their military bearing.

You choose which action to perform to overcome an obstacle, by describing what your character does. Actions that are poorly suited to the situation may be less effective and may put the character in more danger, but they can still be attempted. Usually, when you perform an action, you’ll make an \textbf{action roll} to see how it turns out.

\section{Action Roll}

You make an \textbf{action roll} when your character does something potentially dangerous or troublesome. The possible results of the action roll depend on your character’s \textbf{position}. There are three positions: \textbf{controlled}, \textbf{risky}, and \textbf{desperate}. If you’re in a \textbf{controlled} position, the possible consequences are less serious. If you’re in a \textbf{desperate} position, the consequences can be severe. If you’re somewhere in between, it’s \textbf{risky}---usually considered the “default” position for most actions.

If there’s no danger or trouble at hand, you don’t make an action roll. You might make a \textbf{fortune} roll or a \textbf{downtime} roll or the GM will simply say yes---and you accomplish your goal.

\section{Attribute Ratings}

There are three \textbf{attributes} in the game system that the player characters use to resist bad consequences: \gameterm{Insight} , \gameterm{Prowess} , and \gameterm{Resolve} . Each attribute has a rating (from zero to 4) that tells you how many dice to roll when you use that attribute.

The rating for each attribute is equal to the number of dots in the \textbf{first column} under that attribute (see the examples, at right). The more well-rounded your character is with a particular set of actions, the better their attribute rating.

\section{Resistance Roll}

Each attribute resists a different type of danger. If you get stabbed, for example, you resist physical harm with your \gameterm{Prowess}  rating. Resistance rolls always succeed---you diminish or deflect the bad result---but the better your roll, the less \textbf{stress} it costs to reduce or avoid the danger.

When the enemy has a big advantage, you’ll need to make a resistance roll before you can take your own action. For example, when you duel the master sword-fighter, she disarms you before you can strike. You need to make a resistance roll to keep hold of your blade if you want to attack her. Or perhaps you face a powerful ghost and attempt to \gameterm{Attune}  with it to control its actions. But before you can make your own roll, you must resist possession from the spirit.

The GM judges the threat level of the enemies and uses these “preemptive” resistance rolls as needed to reflect the capabilities of especially dangerous foes.

For details on \textbf{Resistance Rolls}, see page 28.

@TODO Add table from page 9.

\section{Actions}

When you \gameterm{Attune} , you open your mind to arcane power.

\emph{You might communicate with a ghost. You could try to perceive beyond sight in order to better understand your situation (but Surveying might be better).}

When you \gameterm{Command} , you compel swift obedience.

\emph{You might intimidate or threaten to get what you want. You might lead a gang in a group action. You could try to order people around to persuade them (but Consorting might be better).}

When you \gameterm{Consort} , you socialize with friends and contacts.

\emph{You might gain access to resources, information, people, or places. You might make a good impression or win someone over with your charm and style. You might make new friends or connect with your heritage or background. You could try to manipulate your friends with social pressure (but Sway might be better).}

When you \gameterm{Finesse} , you employ dextrous manipulation or subtle misdirection.

\emph{You might pick someone’s pocket. You might handle the controls of a vehicle or direct a mount. You might formally duel an opponent with graceful fighting arts. You could try to employ those arts in a chaotic melee (but Skirmishing might be better). You could try to pick a lock (but Tinkering might be better).}

When you \gameterm{Hunt} , you carefully track a target.

\emph{You might follow a target or discover their location. You might arrange an ambush. You might attack with precision shooting from a distance. You could try to bring your guns to bear in a melee (but Skirmishing might be better).}

When you \gameterm{Prowl} , you traverse skillfully and quietly.

\emph{You might sneak past a guard or hide in the shadows. You might run and leap across the rooftops. You might attack someone from hiding with a back-stab or blackjack. You could try to waylay a victim in the midst of battle (but Skirmishing might be better).}

When you \gameterm{Skirmish} , you entangle a target in close combat so they can’t easily escape.

\emph{You might brawl or wrestle with them. You might hack and slash. You might seize or hold a position in battle. You could try to fight in a formal duel (but Finessing might be better).}

When you \gameterm{Study} , you scrutinize details and interpret evidence.

\emph{You might gather information from documents, newspapers, and books. You might do research on an esoteric topic. You might closely analyze a person to detect lies or true feelings. You could try to examine events to understand a pressing situation (but Surveying might be better).}

When you \gameterm{Survey} , you observe the situation and anticipate outcomes.

\emph{You might spot telltale signs of trouble before it happens. You might uncover opportunities or weaknesses. You might detect a person’s motivations or intentions. You could try to spot a good ambush point (but Hunting might be better).}

When you \gameterm{Sway} , you influence with guile, charm, or argument.

\emph{You might lie convincingly. You might persuade someone to do what you want. You might argue a compelling case that leaves no clear rebuttal. You could try to trick people into affection or obedience (but Consorting or Commanding might be better).}

When you \gameterm{Tinker} , you fiddle with devices and mechanisms.

\emph{You might create a new gadget or alter an existing item. You might pick a lock or crack a safe. You might disable an alarm or trap. You might turn the clockwork and electroplasmic devices around the city to your advantage. You could try to use your technical expertise to control a vehicle (but Finessing might be better).}

When you \gameterm{Wreck} , you unleash savage force.

\emph{You might smash down a door or wall with a sledgehammer, or use an explosive to do the same. You might employ chaos or sabotage to create a distraction or overcome an obstacle. You could try to overwhelm an enemy with sheer force in battle (but Skirmishing might be better).}

As you can see, many actions overlap with others. This is by design. As a player, you get to choose which action you roll, by saying what your character does. Can you try to \gameterm{Wreck}  someone during a fight? Sure! The GM tells you the position and effect level of your action in this circumstance. As it says, \gameterm{Skirmish}  \emph{might} be better (less risky or more effective), depending on the situation at hand (sometimes it won’t be better).

\chapter{Stress \& Trauma}

\section{Stress}

Player characters in Blades in the Dark have a special reserve called \textbf{stress}. When they suffer a consequence that they don’t want to accept, they can take stress instead. The result of the \textbf{resistance roll} determines how much stress it costs to avoid a bad outcome.

\emph{During a knife fight, Daniel’s character, Cross, gets stabbed in the chest. Daniel rolls his \gameterm{Prowess}  rating to resist, and gets a \gameterm{2} . It costs 6 stress, minus 2 (the result of the resistance roll) to resist the consequences. Daniel marks off 4 stress and describes how Cross survives.}

\emph{The GM rules that the harm is reduced by the resistance roll, but not avoided entirely. Cross suffers level 2 harm (“Chest Wound”) instead of level 3 harm (“Punctured Lung”).}

\section{Pushing Yourself}

You can use stress to push yourself for greater performance. For each bonus you choose below, take \textbf{2 stress} (each can be chosen once for a given action):

\begin{itemize}
	\item Add \textbf{+1d} to your roll. (This may be used for an action roll or downtime roll or any other kind of roll where extra effort would help you)
	\item Add \textbf{+1 level} to your effect.
	\item Take action when you’re incapacitated.
\end{itemize}

\section{Trauma}

When a PC marks their last stress box, they suffer a level of \gameterm{trauma} . When you take \gameterm{trauma} , circle one of your \textbf{trauma conditions} like \emph{Cold}, \emph{Reckless}, \emph{Unstable}, etc. They’re all described on the next page.

When you suffer \gameterm{trauma} , you’re taken out of action. You’re “left for dead” or otherwise dropped out of the current conflict, only to come back later, shaken and drained. When you return, \textbf{you have zero stress} and your vice has been satisfied for the next downtime.

\textbf{Trauma conditions are permanent}. Your character acquires the new personality quirk indicated by the condition, and can earn xp by using it to cause trouble. \textbf{When you mark your fourth trauma condition}, your character cannot continue as a daring scoundrel. You must retire them to a different life or send them to prison to take the fall for the crew’s \gameterm{wanted level} .

\subsection{Trauma Conditions}

\begin{itemize}
	\item \gameterm{Cold} : You’re not moved by emotional appeals or social bonds.
	\item \gameterm{Haunted} : You’re often lost in reverie, reliving past horrors, seeing things.
	\item \gameterm{Obsessed} : You’re enthralled by one thing: an activity, a person, an ideology.
	\item \gameterm{Paranoid} : You imagine danger everywhere; you can’t trust others.
	\item \gameterm{Reckless} : You have little regard for your own safety or best interests.
	\item \gameterm{Soft} : You lose your edge; you become sentimental, passive, gentle.
	\item \gameterm{Unstable} : Your emotional state is volatile. You can instantly rage, or fall into despair, act impulsively, or freeze up.
	\item \gameterm{Vicious} : You seek out opportunities to hurt people, even for no good reason.
\end{itemize}

\chapter{Progress clocks}

A \textbf{progress clock} is a circle divided into segments (see examples at right). Draw a progress clock when you need to track ongoing effort against an obstacle or the approach of impending trouble.

@TODO add clock illustration

\emph{Sneaking into the constables watch tower? Make a clock to track the alert level of the patrolling guards. When the PCs suffer consequences from partial successes or missed rolls, fill in segments on the clock until the alarm is raised.}

Generally, the more complex the problem, the more segments in the progress clock.

A complex obstacle is a 4-segment clock. A more complicated obstacle is a 6-clock. A daunting obstacle is an 8-segment clock.

When you create a clock, make it about the \textbf{obstacle,} not the method. The clocks for an infiltration should be “Interior Patrols” and “The Tower,” not “Sneak Past the Guards” or “Climb the Tower.” The patrols and the tower are the obstacles---the PCs can attempt to overcome them in a variety of ways.

Complex enemy threats can be broken into several “layers,” each with its own progress clock. For example, the dockside gangs’ HQ might have a “Perimeter Security” clock, an “Interior Guards” clock, and a “Office Security” clock. The crew would have to make their way through all three layers to reach the gang boss’ personal safe and valuables within.

Remember that a clock tracks progress. It reflects the fictional situation, so the group can gauge how they’re doing. A clock is like a speedometer in a car. It \emph{shows} the speed of the vehicle---it doesn’t determine the speed.

\section{Simple Obstacles}

Not every situation and obstacle requires a clock. Use clocks when a situation is complex or layered and you need to track something over time---otherwise, resolve the result of an action with a single roll.

Examples of progress clocks follow.

\section{Danger Clocks}

The GM can use a clock to represent a progressive danger, like suspicion growing during a seduction, the proximity of pursuers in a chase, or the alert level of guards on patrol. In this case, when a complication occurs, the GM ticks one, two, or three segments on the clock, depending on the consequence level. When the clock is full, the danger comes to fruition---the guards hunt down the intruders, activate an alarm, release the hounds, etc.

\section{Racing Clocks}

Create two opposed clocks to represent a race. The PCs might have a progress clock called “Escape” while the constables have a clock called “Cornered.” If the PCs finish their clock before the constables fill theirs, they get away. Otherwise, they’re cornered and can’t flee. If both complete at the same time, the PCs escape to their lair, but the hunting officers are outside!

You can also use racing clocks for an environmental hazard. Maybe the PCs are trying to complete the “Search” clock to find the lockbox on the sinking ship before the GM fills the “Sunk” clock and the vessel goes down.

\section{Linked Clocks}

You can make a clock that unlocks another clock once it’s filled. For example, the GM might make a linked clock called “Trapped” after an “Alert” clock fills up. When you fight a veteran warrior, she might have a clock for her “Defense” and then a linked clock for “Vulnerable.” Once you overcome the “Defense” clock, then you can attempt to overcome the “Vulnerable” clock and defeat her.  You might affect the “Defense” clock with violence in a knife-fight, or you lower her defense with deception if you have the opportunity. As always, the method of action is up to the players and the details of the fiction at hand.

\section{Mission Clocks}

The GM can make a clock for a time-sensitive mission, to represent the window of opportunity you have to complete it. If the countdown runs out, the mission is scrubbed or changes---the target escapes, the household wakes up for the day, etc.

\section{Tug-of-war Clocks}

You can make a clock that can be filled \emph{and} emptied by events, to represent a back-and-forth situation. You might make a “Revolution!” clock that indicates when the refugees start to riot over poor treatment. Some events will tick the clock up and some will tick it down. Once it fills, the revolution begins. A tug-of-war clock is also perfect for an ongoing turf war between two crews or factions.

\section{Long-term Project}

Some projects will take a long time. A basic long-term project (like tinkering up a new feature for a device) is eight segments. Truly long-term projects (like creating a new designer drug) can be two, three, or even four clocks, representing all the phases of development, testing, and final completion. Add or subtract clocks depending on the details of the situation and complexity of the project.

A long-term project is a good catch-all for dealing with any unusual player goal, including things that circumvent or change elements of the mechanics or the setting.

\section{Faction Clocks}

Each faction has a long-term goal. When the PCs have \textbf{downtime}, the GM ticks forward the faction clocks that they’re interested in. In this way, the world around the PCs is dynamic and things happen that they’re not directly connected to, changing the overall situation in the city and creating new opportunities and challenges.

The PCs may also directly affect NPC faction clocks, based on the missions and scores they pull off. Discuss known faction projects that they might aid or interfere with, and also consider how a PC operation might affect the NPC clocks, whether the players intended it or not.

\chapter{Action Roll}

When a player character does something challenging, we make an \textbf{action roll} to see how it turns out. An action is challenging if there’s an obstacle to the PC’s goal that’s dangerous or troublesome in some way. We don’t make an action roll unless the PC is put to the test. If their action is something that we’d expect them to simply accomplish, then we don’t make an action roll.

\emph{Each game group will have their own ideas about what “challenging” means. This is good! It’s something that establishes the tone and style of your Blades series.}

To make an action roll, we go through six steps. In play, they flow together somewhat, but let’s break each one down here for clarity.

\begin{itemize}
	\item The player states their \textbf{goal} for the action.
	\item The player chooses the \textbf{action rating}.
	\item The GM sets the \textbf{position} for the roll.
	\item The GM sets the \textbf{effect level} for the action.
	\item Add \textbf{bonus dice}.
	\item The player rolls the \textbf{dice} and we judge the result.
\end{itemize}

\section{1. The Player States Their Goal}

Your goal is the concrete outcome your character will achieve when they overcome the obstacle at hand. Usually the character’s goal is pretty obvious in context, but it’s the GM’s job to ask and clarify the goal when necessary.

\emph{“You’re punching him in the face, right? Okay... what do want to get out of this? Do you want to take him out, or just rough him up so he’ll do what you want?”}

\section{2. The Player Chooses the Action Rating}

The player chooses which \textbf{action rating} to roll, following from what their character is doing on-screen. If you want to roll your \gameterm{Skirmish}  action, then get in a fight. If you want to roll your \gameterm{Command}  action, then order someone around. You can’t roll a given action rating unless your character is presently performing that action in the fiction.

\section{3. The GM Sets the Position}

Once the player chooses their action, the GM sets the \textbf{position} for the roll. The position represents how dangerous or troublesome the action might be. There are three positions: \textbf{controlled}, \textbf{risky}, and \textbf{desperate}. To choose a position, the GM looks at the profiles for the positions below and picks one that most closely matches the situation at hand.

\textbf{By default, an action roll is risky.} You wouldn’t be rolling if there was no risk involved. If the situation seems more dangerous, make it desperate. If it seems less dangerous, make it controlled.

\section{4. The GM Sets the Effect Level}

The GM assesses the likely \textbf{effect level} of this action, given the factors of the situation. Essentially, the effect level tells us “how much” this action can accomplish: will it have \textbf{limited}, \textbf{standard}, or \textbf{great} effect?

\emph{The GM’s choices for effect level and position can be strongly influenced by the player’s choice of action rating. If a player wants to try to make a new friend by \gameterm{Wrecking}  something---well... maybe that’s possible, but the GM wouldn’t be crazy to say it’s a desperate roll and probably limited effect. Seems like \gameterm{Consorting}  would be a lot better for that. The players are always free to choose the action they perform, but that doesn’t mean all actions should be equally risky or potent.}

\section{5. Add Bonus Dice}

You can normally get two bonus dice for your action roll (some special abilities might give you additional bonus dice).

For one bonus die, you can get \textbf{assistance} from a teammate. They take 1 stress, say how they help you, and give you +1d.

For another bonus die, you can either \textbf{push yourself} (take 2 stress) or you can accept a \textbf{Devil’s Bargain} (you can’t get dice for both, it’s one or the other).

\subsection{The Devil’s Bargain}

PCs in \emph{Blades} are reckless scoundrels addicted to destructive vices---they don’t always act in their own best interests. To reflect this, the GM or any other player can offer you a bonus die if you accept a Devil’s Bargain. Common Devil’s Bargains include:

\begin{itemize}
	\item Collateral damage, unintended harm.
	\item Sacrifice \gameterm{coin}  or an item.
	\item Betray a friend or loved one.
	\item Offend or anger a faction.
	\item Start and/or tick a troublesome clock.
	\item Add \gameterm{heat}  to the crew from evidence or witnesses.
	\item Suffer harm.
\end{itemize}

The Devil’s Bargain occurs regardless of the outcome of the roll. You make the deal, pay the price, and get the bonus die.

The Devil’s Bargain is always a free choice. If you don’t like one, just reject it (or suggest how to alter it so you might consider taking it). You can always just push yourself for that bonus die instead.

If it’s ever needed, the GM has final say over which Devil’s Bargains are valid.

\section{6. Roll the Dice and Judge the Result}

Once the goal, action rating, position, and effect have been established, add any bonus dice and roll the dice pool to determine the outcome. (See the sets of possible outcomes, by position, on the next page.)

The action roll does a lot of work for you. It tells you how well the character performs as well as how serious the consequences are for them. They might succeed at their action without any consequences (on a \gameterm{6} ), or they might succeed but suffer consequences (on a \gameterm{4/5} ), or it might just all go wrong (on a \gameterm{1-3} ).

On a \gameterm{1-3} , it’s up to the GM to decide if the PC’s action has any effect or not, or if it even happens at all. Usually, the action just fails completely, but in some circumstances, it might make sense or be more interesting for the action to have some effect even on a \gameterm{1-3}  result.

Each \gameterm{4/5}  and \gameterm{1-3}  outcome lists suggested \textbf{consequences} for the character. The worse your position, the worse the consequences are. The GM can inflict one or more of these consequences, depending on the circumstances of the action roll. PCs have the ability to avoid or reduce the severity of consequences that they suffer by \textbf{resisting} them.

When you narrate the action after the roll, the GM and player collaborate together to say what happens on-screen. \emph{Tell us how you vault across to the other rooftop. Tell us what you say to the Inspector to convince her. The GM will tell us how she reacts. When you face the Red Sash duelist, what’s your fighting style like? Etc.}

\section{Action Roll Summary}

\begin{itemize}
	\item A player or GM calls for a roll. Make an \textbf{action roll} when the character performs a dangerous or troublesome action.
	\item The player chooses the \textbf{action rating} to roll. Choose the action that matches what the character is doing in the fiction.
	\item The GM establishes the \textbf{position} and \textbf{effect level} of the action. The choice of position and effect is influenced strongly by the player’s choice of action.
	\item Add up to two bonus dice. 1) \textbf{Assistance} from a teammate. 2) \textbf{Push yourself} (take 2 stress) or accept a \textbf{Devil’s Bargain}.
	\item Roll the dice pool and judge the outcome. The players and GM narrate the action together. The GM has final say over what happens and inflicts consequences as called for by the position and the result of the roll.
\end{itemize}

\section{Double-duty Rolls}

Since NPCs don’t roll for their actions, an action roll does double-duty: \textbf{it resolves the action of the PC as well as any NPCs that are involved}. The single roll tells us how those actions interact and which consequences result. On a \gameterm{6} , the PC wins and has their effect. On a \gameterm{4/5} , it’s a mix---both the PC and the NPC have their effect. On a \gameterm{1-3} , the NPC wins and has their effect as a consequence on the PC.

@TODO add Action Roll table p21

\chapter{Effect}

In Blades in the Dark, you achieve goals by taking actions and facing consequences. But how many actions does it take to achieve a particular goal? That depends on the \textbf{effect level} of your actions. The GM judges the effect level using the profiles below. Which one best matches the action at hand---\textbf{great}, \textbf{standard}, or \textbf{limited}? Each effect level indicates the questions that should be answered for that effect, as well as how many segments to tick if you’re using a \textbf{progress clock}.

@TODO add fancy effect levels p23.

\section{Assessing Factors}

To assess effect level, first start with your gut feeling, given this situation. Then, if needed, assess three factors that may modify the effect level: \textbf{potency}, \textbf{scale}, and \textbf{quality}. If the PC has an advantage in a given factor, consider a higher effect level. If they have a disadvantage, consider a reduced effect level.

\subsection{Potency}

The potency factor considers particular weaknesses, taking extra time or a bigger risk, or the influence of arcane powers. An infiltrator is more potent if all the lights are extinguished and they move about in the dark.

\subsection{Quality/Tier}

Quality represents the effectiveness of tools, weapons, or other resources, usually summarized by Tier. \textbf{Fine items} count as +1 bonus in quality, stacking with Tier.

\emph{Thorn is picking the lock to a safehouse run of a gang reknowned for Occult dealings. Her crew is Tier I and she has fine lockpicks---so she’s effectively Tier II. The Occult gang is Tier III. Thorn is outclassed in quality, so her effect will be limited on the lock.}

\subsection{Scale}

Scale represents the number of opponents, size of an area covered, scope of influence, etc. Larger scale can be an advantage or disadvantage depending on the situation. In battle, more people are better. When infiltrating, more people are a hindrance.

When considering factors, effect level might be reduced below limited, resulting in \textbf{zero effect}---or increased beyond great, resulting in an \textbf{extreme effect}.

If a PC special ability gives “+1 effect,” it comes into play \emph{after} the GM has assessed the effect level. For example, if you ended up with zero effect, the +1 effect bonus from your Cutter’s \gameterm{Bodyguard}  ability would bump them up to limited effect.

Also, remember that a PC can \textbf{push themselves} (take 2 stress) to get +1 effect on their action.

Every factor won’t always apply to every situation. You don’t have to do an exact accounting every time, either. Use the factors to help you make a stronger judgment call---don’t feel beholden to them.

\section{Trading Position for Effect}

After factors are considered and the GM has announced the effect level, a player might want to trade position for effect, or vice versa. For instance, if they’re going to make a risky roll with standard effect (the most common scenario, generally), they might instead want to push their luck and make a desperate roll but with great effect.

This kind of trade-off isn’t included in the effect factors because it’s not an element the GM should assess when setting the effect level. Once the level is set, though, you can always offer the trade-off to the player if it makes sense in the situation.

\emph{“I Prowl across the courtyard and vault over the wall, hiding in the shadows by the canal dock and gondola.”}

\emph{“I don’t think you can make it across in one quick dash. The scale of the courtyard is a factor here, so your effect will be limited. Let’s say you can get halfway across with this action, then you’ll have to Prowl through the other half of the space (and the rest of the guards there) to reach the other side.”}

\emph{“I didn’t realize it was that far. Hmmm. Okay, what if I just go as fast as I can. Can I get all the way across if I make a desperate roll?”}

\emph{“Yep, sounds good to me!”}

\section{Consequences}

When a PC suffers an effect from an enemy or a dangerous situation, it’s called a \textbf{consequence}. Consequences are the companion to effects. PCs have effect on the world around them and they suffer consequences in return from the risks they face.

\chapter{Setting Position \& Effect}

The GM sets position and effect for an action roll at the same time, after the player says what they’re doing and chooses their action. Usually, \textbf{Risky / Standard} is the default combination, modified by the action being used, the strength of the opposition, and the effect factors.

The ability to set position and effect as independent variables gives you nine combinations to choose from, to help you convey a wide array of fictional circumstances.

\emph{For example, if a character is facing off alone against a small enemy gang, the situation might be:}

\begin{itemize}
	\item \emph{She fights the gang straight up, rushing into their midst, hacking away in a wild }\gameterm{Skirmish} \emph{. In this case, being threatened by the larger force lowers her position to indicate greater risk, and the scale of the gang reduces her effect (Desperate / Limited).}
	\item \emph{She fights the gang from a choke-point, like a narrow alleyway where their numbers can’t overwhelm her at once. She’s not threatened by several at once, so her risk is similar to a one-on-one fight, but there’s still a lot of enemies to deal with, so her effect is reduced (Risky / Limited).}
	\item \emph{She doesn’t fight the gang, instead trying to maneuver her way past them and escape. She’s still under threat from many enemy attacks, so her position is worse, but if the ground is open and the gang can’t easily corral her, then her effect for escaping isn’t reduced (Desperate / Standard). If she had some immediate means of escape (like leaping onto a speeding carriage), then her effect might even be increased (Desperate / Great).}
	\item \emph{The gang isn’t aware of her yet---she’s set up in a sniper position on a nearby roof. She takes a shot against one of them. Their greater numbers aren’t a factor, so her effect isn’t reduced, and she’s not immediately in any danger (Controlled / Great). Maybe instead she wants to fire off a salvo of suppressing fire against the whole gang, in which case their scale applies (Controlled / Limited). If the gang is on guard for potential trouble, her position is more dangerous (Risky / Great). If the gang is alerted to a sniper, then the effect may be reduced further, as they scatter and take cover (Risky / Limited). If the gang is able to muster covering fire while they fall back to a safe position, then things are even worse for our scoundrel (Desperate / Limited).}
\end{itemize}

\chapter{Consequences and Harm}

Enemy actions, bad circumstances, or the outcome of a roll can inflict \textbf{consequences} on a PC. There are five types (at right).

A given circumstance might result in one or more consequences, depending on the situation. The GM determines the consequences, following from the fiction and the style and tone established by the game group.

\section{Reduced Effect}

This consequence represents impaired performance. The PC’s action isn’t as effective as they’d anticipated. You hit him, but it’s only a flesh wound. She accepts the forged invitation, but she’ll keep her eye on you throughout the night. You’re able to scale the wall, but it’s slow going---you’re only halfway up. This consequence essentially reduces the effect level of the PC’s action by one after all other factors are accounted for.

\section{Complication}

This consequence represents trouble, mounting danger, or a new threat. The GM might introduce an immediate problem that results from the action right now: the room catches fire, you’re disarmed, the crew takes +1 \gameterm{heat}  from evidence or witnesses, you lose status with a faction, the target evades you and now it’s a chase, reinforcements arrive, etc.

Or the GM might tick a clock for the complication, instead. Maybe there’s a clock for the alert level of the guards at the manor. Or maybe the GM creates a new clock for the suspicion of the noble guests at the masquerade party and ticks it. Fill one tick on a clock for a minor complication or two ticks for a standard complication.

A \textbf{serious complication} is more severe: reinforcements surround and trap you, the room catches fire and falling ceiling beams block the door, your weapon is broken, the crew suffers +2 \gameterm{heat} , your target escapes out of sight, etc. Fill three ticks on a clock for a serious complication.

\textbf{Don’t inflict a complication that negates a successful roll.} If a PC tries to corner an enemy and gets a \gameterm{4/5} , don’t say that the enemy escapes. The player’s roll succeeded, so the enemy is cornered... maybe the PC has to wrestle them into position and during the scuffle the enemy grabs their gun.

\section{Lost Opportunity}

This consequence represents shifting circumstance. You had an opportunity to achieve your goal with this action, but it slips away. To try again, you need a new approach---usually a new form of action or a change in circumstances. Maybe you tried to \gameterm{Skirmish}  with the noble to trap her on the balcony, but she evades your maneuver and leaps out of reach. If you want to trap her now you’ll have to try another way---maybe by \gameterm{Swaying}  her with your roguish charm.

\section{Worse Position}

This consequence represents losing control of the situation---the action carries you into a more dangerous position. Perhaps you make the leap across to the next rooftop, only to end up dangling by your fingertips. You haven’t failed, but you haven’t succeeded yet, either. You can try again, re-rolling at the new, worse position. This is a good consequence to choose to show escalating action. A situation might go from controlled, to risky, to desperate as the action plays out and the PC gets deeper and deeper in trouble.

\section{Harm}

This consequence represents a long-lasting debility (or death). When you suffer harm, record the specific injury on your character sheet equal to the level of harm you suffer. If you suffer\textbf{ lesser harm}, record it in the bottom row. If you suffer \textbf{moderate harm}, write it in the middle row. If you suffer \textbf{severe harm}, record it in the top row. See examples of harm and the harm tracker, below.

Your character suffers the penalty indicated at the end of the row if any or all harm recorded in that row applies to the situation at hand. So, if you have “Drained” and “Battered” harm in the bottom row, you’ll suffer reduced effect when you try to run away from the constables. When you’re impaired by harm in the top row (severe harm, level 3), your character is incapacitated and can’t do anything unless you have help from someone else or \textbf{push yourself} to perform the action.

If you need to mark a harm level, but the row is already filled, the harm moves up to the next row above. So, if you suffered standard harm (level 2) but had no empty spaces in the second row, you’d have to record severe harm (level 3), instead. If you run out of spaces on the top row and need to mark harm there, your character suffers a \textbf{catastrophic, permanent consequence} (loss of a limb, sudden death, etc., depending on the circumstances).

@TODO Table from p27


\subsection{Harm examples}

\textbf{Fatal (4):} \emph{Electrocuted, Drowned, Stabbed in the Heart.}

\textbf{Severe (3):} \emph{Impaled, Broken Leg, Shot in Chest, Badly Burned, Terrified.}

\textbf{Moderate (2):} \emph{Exhausted, Deep Cut to Arm, Concussion, Panicked, Seduced.}

\textbf{Lesser (1):} \emph{Battered, Drained, Distracted, Scared, Confused.}

Harm like “Drained” or “Exhausted” can be a good fallback consequence if there’s nothing else threatening a PC (like when they spend all night \gameterm{Studying}  those old books, looking for any clues to their enemy’s weaknesses before he strikes).

\chapter{Resistance \& Armor}

When your PC suffers a consequence that you don’t like, you can choose to resist it. Just tell the GM, “No, I don’t think so. I’m resisting that.” Resistance is always automatically effective---the GM will tell you if the consequence is reduced in severity or if you avoid it entirely. Then, you’ll make a \textbf{resistance roll} to see how much stress your character suffers as a result of their resistance.

You make the roll using one of your character’s \textbf{attributes} (\gameterm{Insight} , \gameterm{Prowess} , or \gameterm{Resolve} ). The GM chooses the attribute, based on the nature of consequences:

\begin{itemize}
	\item \gameterm{Insight} : Consequences from deception or understanding.
	\item \gameterm{Prowess} : Consequences from physical strain or injury.
	\item \gameterm{Resolve} : Consequences from mental strain or willpower.
\end{itemize}

Your character suffers \textbf{6 stress} when they resist, \textbf{minus the highest die result from the resistance roll}. So, if you rolled a \gameterm{4} , you’d suffer 2 stress. If you rolled a \gameterm{6} , you’d suffer zero stress. If you get a \gameterm{critical}  result, you also \textbf{clear 1 stress}.

\begin{quote}
	Ian’s character, Silas, is in a desperate \gameterm{Skirmish}  with several duelists and one of them lands a blow with their sword. Since the position was desperate, the GM inflicts severe harm (modified by any other factors). They tell Ian to record level 3 harm, “Gut Stabbed” on Silas’s sheet. Ian decides to resist the harm, instead. The GM says he can reduce the harm by one level if he resists it. Ian rolls 3d for Silas’s \gameterm{Prowess}  attribute and gets a \gameterm{5} . Silas takes 1 stress and the harm is reduced to level 2, “Cut to the Ribs.”
\end{quote}

Usually, a resistance roll \textbf{reduces the severity} of a consequence. If you’re going to suffer fatal harm, for example, a resistance roll would reduce the harm to severe, instead. Or if you got a complication when you were sneaking into the manor house, and the GM was going to mark three ticks on the “Alert” clock, she’d only mark two (or maybe one) if you resisted the complication.

\textbf{You may only roll against a given consequence once.}

The GM also has the option to rule that your character \textbf{completely avoids} the consequence. For instance, maybe you’re in a sword fight and the consequence is getting disarmed. When you resist, the GM says that you avoid that consequence completely: you keep hold of your weapon.

\textbf{By adjusting which consequences are reduced vs. which are avoided, the GM establishes the overall tone of your game}. For a more daring game, most consequences will be avoided. For a grittier game, most consequences will only be reduced with resistance.

The GM may also threaten several consequences at once, then the player may choose which ones to resist (and make rolls for each).

\begin{quote}
	“She stabs you and then leaps off the balcony. Level 2 harm and you lose the opportunity to catch her with fighting.”
\end{quote}

\begin{quote}
	“I’ll resist losing the opportunity by grappling her as she attacks. She can stab me, but I don’t want to let her escape.”
\end{quote}

Once you decide to resist a consequence and roll, you suffer the stress indicated. You can’t roll first and see how much stress you’ll take, then decide whether or not to resist.

@TODO add table p29

\section{Armor}

If you have a type of \textbf{armor} that applies to the situation, you can mark an armor box to reduce or avoid a consequence, instead of rolling to resist.

\begin{quote}
Silas is taking level 2 harm, “Cut to the Ribs,” and the fight isn’t even over yet, so Ian decides to use Silas’s armor to reduce the harm. He marks the armor box and the harm becomes level 1, “Bruised.” If Silas was wearing heavy armor, he could mark a second armor box and reduce the harm again, to zero.
\end{quote}

When an armor box is marked, it can’t be used again until it’s restored. All of your armor is restored when you choose your \textbf{load} for the next score.

\section{Death}

There are a couple ways for a PC to die:

\begin{itemize}
	\item If they suffer level 4 fatal harm and they don’t resist it, they die. \emph{Sometimes this is a choice a player wants to make, because they feel like it wouldn’t make sense for the character to survive or it seems right for their character to die here.}
	\item If they need to record harm at level 3 and it’s already filled, they suffer a catastrophic consequence, which might mean sudden death (depending on the circumstances).
\end{itemize}

When your character dies, you have options:

\begin{itemize}
	\item You can create a new scoundrel to play. Maybe you “promote” one of the NPC gang members to a PC, or create a brand new character who joins the crew.
\end{itemize}

\chapter{Fortune Roll}

The fortune roll is a tool the GM can use to disclaim decision making. You use a fortune roll in two different ways:

\textbf{When you need to make a determination about a situation the PCs aren’t directly involved in} and don’t want to simply decide the outcome.

\begin{quote}
	Two rival gangs are fighting. How does that turn out? The GM makes a fortune roll for each of them. One gets a good result but the other gets limited effect. The GM decides that the first gang takes over some of their rivals’ turf but suffer some injuries during the skirmish.
\end{quote}

\textbf{When an outcome is uncertain}, but no other roll applies to the situation at hand.

\begin{quote}
	While pilfering the workshop of an alchemist, Cross is possessed by a vengeful ghost. As control of his body slips away, Cross grabs a random potion bottle and drinks it down. Will the arcane concoction have an effect on the spirit? Will it poison Nock to death? Who knows? The GM makes a fortune roll to see how it turns out.
\end{quote}

When you make a fortune roll you may assess \textbf{any trait rating} to determine the dice pool of the roll.

\begin{itemize}
	\item When a faction takes an action with uncertain outcome, you might use their \textbf{Tier} rating to make a fortune roll.
	\item When a gang operates independently, use their \textbf{quality} rating for a fortune roll.
	\item When a supernatural power manifests with uncertain results, you might use its \textbf{magnitude} for a fortune roll.
	\item When a PC \textbf{gathers information}, you might make a fortune roll using their \textbf{action rating} to determine the amount of the info they get.
\end{itemize}

If no trait applies, roll 1d for sheer luck or create a dice pool (from one to four) based on the situation at hand. If two parties are directly opposed, make a fortune roll for each side to see how they do, then assess the outcome of the situation by comparing their performance levels.

The fortune roll is also a good tool to help the GM manage all the various moving parts of the world. Sometimes a quick roll is enough to answer a question or inspire an idea for what might happen next.

Other examples of fortune rolls:

\begin{itemize}
	\item The PCs instigate a war between two factions, then sit back and watch the fireworks. How does it turn out? Does either side dominate? Are they both made vulnerable by the conflict? Make a few fortune rolls to find out.
	\item A strange sickness is sweeping the city. How badly is a crime ridden district hit by the outbreak? The GM assigns a magnitude to the arcane plague, and makes a fortune roll to judge the extent of its contamination.
	\item The Hound stakes out a good spot and makes a sniper shot against a gang leader when he enters his office. The controlled \gameterm{Hunt}  roll is a success, but is great effect enough to instantly kill a grizzled gang leader? Instead of making a progress clock for his mortality, the GM decides to use a simple fortune roll with his “toughness” as a trait to see if he can possibly survive the attack. The roll is a \gameterm{4/5} : the bullet misses his heart, but hits him in the lung---it’s a mortal wound. He’s on death’s door, with only hours to live, unless his gang can get an expert physicker to him in time.
	\item Inspectors are putting a case together against the PC crew. How quickly will their evidence result in arrests? The crew’s \gameterm{wanted level}  counts as a major advantage for the inspectors.
	\item The PCs face off in a skirmish with a veteran demon hunter captain and her crew. The tide of battle goes in the PCs’ favor, and many crew members are killed. One of the players asks if the captain will surrender to spare the rest of her crew’s lives. The GM isn’t sure. How cold-hearted is this veteran hunter? She’s stared giant demons in the eye without flinching... is there anything human left inside her? The GM makes a 2d fortune roll for “human feelings” to see if a spark of compassion remains in heart. If so, maybe one of the PCs can roll to \gameterm{Consort} , \gameterm{Sway} , or \gameterm{Command}  her to stand down.
\end{itemize}

@TODO table p31

\chapter{Gathering Information}

The flow of information from the GM to the players about the fictional world is very important in a roleplaying game. By default, the GM tells the players what their characters perceive, suspect, and intuit. But there’s just too much going on to say \emph{everything}---it would take forever and be boring, too. The players have a tool at their disposal to more fully investigate the fictional world.

When you want to know something specific about the fictional world, your character can \textbf{gather information}. The GM will ask you \textbf{how} your character gathers the info (or how they learned it in the past).

If it’s common knowledge, the GM will simply answer your questions. If there’s an obstacle to the discovery of the answer, an action roll is called for. If it’s not common knowledge but there’s no obstacle, a simple fortune roll determines the quality of the information you gather.

Each attempt to gather information takes time. If the situation allows, you can try again if you don’t initially get all the info that you want. But often, the opportunity is fleeting, and you’ll only get one chance to roll for that particular question.

Some example questions are on the bottom of the character sheet. The GM always answers honestly, but with a level of detail according to the level of effect.

The most common gather information actions are \gameterm{Surveying}  the situation to reveal or anticipate what’s going on and \gameterm{Studying}  a person to understand what they intend to do or what they’re really thinking.

Sometimes, you’ll have to maneuver yourself into position before you can gather information. For example, you might have to \gameterm{Prowl}  to a good hiding place first and then \gameterm{Study}  the cultists when they perform their dark ritual.

\section{Investigation}

Some questions are too complex to answer immediately with a single gather information roll. For instance, you might want to discover the network of contraband smuggling routes in the city. In these cases, the GM will tell you to start a \textbf{long-term project} that you work on during \textbf{downtime}.

You track the investigation project using a progress clock. Once the clock is filled, you have the evidence you need to ask several questions about the subject of your investigation as if you had great effect.

\section{Examples \& Questions}

\begin{itemize}
	\item You might \gameterm{Attune}  to see echoes of recent spirit activity. \emph{Have any new ghosts been here? How can I find the spirit well that’s calling to them? What should I be worried about?}
	\item You might \gameterm{Command}  a local barkeep to tell you what he knows about the secret meetings held in his back room. \emph{What’s really going on here? What’s he really feeling about this? Is he part of this secret group?}
	\item You might \gameterm{Consort}  with a well-connected friend to learn secrets about an enemy, rival, or potential ally. \emph{What do they intend to do? What might I suspect about their motives? How can I discover leverage to manipulate them?}
	\item You might \gameterm{Hunt}  a courier across the city, to discover who’s receiving satchels of coin from a master duelsit. \emph{Where does the package end up? How can I find out who signed for the package at City Hall?}
	\item You might \gameterm{Study}  ancient and obscure books to discover an arcane secret. \emph{How can I disable the runes of warding? }\emph{Will anyone sense if they’re disabled?}
	\item Or you might \gameterm{Study}  a person to read their intentions and feelings. \emph{What are they really feeling? How could I get them to trust me?}
	\item You might \gameterm{Survey}  a manor house to case it for a heist. \emph{What’s a good point of infiltration? What’s the danger here?}
	\item Or you might \gameterm{Survey}  a charged situation when you meet another gang. \emph{What’s really going on here? Are they about to attack us?}
	\item You might \gameterm{Sway}  a powerful lord at a party so he divulges his future plans. \emph{What does he intend to do? How can I get him to think I might be a good partner in this venture?}
	\item Or you might \gameterm{Sway}  his bodyguard to confide in you about recent events. \emph{Where has he been lately? Who’s he been meeting with?}
\end{itemize}

@TODO table p33

\chapter{Coin and Stash}

\section{Coin}

\gameterm{Coin}  is an abstract measure of cash and liquid assets.

The few bits PCs use in their daily lives are not tracked. If a character wants to spend to achieve a small goal (bribe a doorman), use the PC’s \textbf{lifestyle quality} for a fortune roll.

\subsection{Monetary values}

\begin{itemize}
	\item \gameterm{1 coin: } A full purse of silver pieces. A week’s wages.
	\item \gameterm{2 coin:}  A fine weapon. A weekly income for a small business. A fine piece of art. A set of luxury clothes.
	\item \gameterm{4 coin:}  A satchel full of silver. A month’s wages.
	\item \gameterm{6 coin:}  An exquisite jewel. A heavy burden of silver pieces.
	\item \gameterm{8 coin:}  A good monthly take for a small business. A small safe full of coins and valuables. A very rare luxury commodity.
	\item \gameterm{10 coin:}  Liquidating a significant asset---a carriage and goats, a horse, a deed to a small property.
\end{itemize}

More than 4\gameterm{coin}  is an impractical amount to keep lying around. You must spend the excess or put it in your \textbf{stash} (see below). A crew can also store 4 \gameterm{coin}  in their lair, by default. If they upgrade to a \textbf{vault}, they can expand their stores to 8 and then 16 \gameterm{coin} . Any \gameterm{coin}  beyond their limit must be spent as soon as possible (typically before the next score) or distributed among the crew members.

One unit of \gameterm{coin}  in silver pieces or other bulk currency takes up one item slot for your \textbf{load} when carried.

\subsection{Coin Use}

\begin{itemize}
	\item Spend 1 \gameterm{coin}  to get an additional \textbf{activity} during \textbf{downtime}.
	\item Spend 1 \gameterm{coin}  to increase the result level of a \textbf{downtime activity} roll.
	\item Spend \gameterm{coin}  to avoid certain crew \textbf{entanglements}.
	\item Put \gameterm{coin}  in your character’s \textbf{stash} to improve their lifestyle and circumstances when they retire. See the next page.
	\item Spend \gameterm{coin}  when you advance your crew’s \textbf{Tier}.
\end{itemize}

\section{Stash \& Retirement}

When you mark your character’s final \gameterm{trauma}  and they retire, the amount of \gameterm{coin}  they’ve managed to stash away determines their fate. Your stash tracker is on your character sheet.

\begin{itemize}
	\item \textbf{Stash 0-10:} \textbf{Poor soul. }You end up in the gutter, awash in vice and misery.
	\item \textbf{Stash 11-20: Meager.} A tiny hovel that you can call your own.
	\item \textbf{Stash 21-39: Modest.} A simple home or apartment, with some small comforts. You might operate a tavern or small business.
	\item \textbf{Stash 40: Fine.} A well-appointed home or apartment, claiming a few luxuries. You might operate a medium business.
\end{itemize}

In addition, each full row of stash (10 \gameterm{coins} ) indicates the \textbf{quality level of the character’s lifestyle}, from zero (street life) to four (luxury).

\begin{quote}
	Cross wants some alone-time with a prospective new friend, but he can’t take them back to the hidden lair where he lives, so what to do? Ryan, Cross’s player, says he wants to rent a nice room for the evening, so the GM asks for a fortune roll using Cross’s lifestyle rating to see what quality of room Cross can manage.
\end{quote}

\subsection{Removing coin from your stash}

If you want to pull \gameterm{coin}  out of your stash, you may do so, at a cost. Your character sells off some of their assets and investments in order to get some quick cash. \textbf{For every 2 stash removed, you get 1 }\gameterm{coin} \textbf{ in cash.}

\chapter{The Faction Game}

\section{Tier}

Each notable faction is ranked by \textbf{Tier}---a measure of wealth, influence, and scale. At the highest level are the Tier V and VI factions, the true powers of the city. Your crew begins at Tier 0.

You’ll use your Tier rating to roll dice when you acquire an asset, as well as for any fortune roll for which your crew’s overall power level and influence is the primary trait. Most importantly, your Tier determines the \textbf{quality level} of your items as well as the quality and \textbf{scale} of the gangs your crew employs---and thereby what size of enemy you can expect to handle.

\subsection{Gang scale by tier}

\begin{itemize}
	\item \gameterm{tier v} . Massive gangs. (80 people)
	\item \gameterm{tier iv} . Huge gangs. (40 people)
	\item \gameterm{tier iii} . Large gangs. (20 people)
	\item \gameterm{tier ii} . Medium gangs. (12 people)
	\item \gameterm{tier i} . Small gangs. (3-6 people)
	\item \gameterm{tier 0} . 1 or 2 people
\end{itemize}

\section{Hold}

On the faction ladder next to the Tier numbers is a letter indicating the strength of each faction’s \textbf{hold}. Hold represents how well a faction can maintain their current position on the ladder. W indicates \textbf{weak} hold. S indicates \textbf{strong} hold. Your crew begins with \textbf{strong} hold at \textbf{Tier 0}.

\section{Development}

To move up the ladder and develop your crew, you need \gameterm{rep} . \gameterm{Rep}  is a measure of clout and renown. When you accrue enough \gameterm{rep} , the other factions take you more seriously and you attract the support needed to develop and grow.

When you complete a score, your crew earns \textbf{2 }\gameterm{rep} . If the target of the score is higher Tier than your crew, you get \textbf{+1 }\gameterm{rep} \textbf{ per Tier higher}. If the target of the score is lower Tier, you get \textbf{-1 \gameterm{rep} } per Tier lower** (minimum zero).

You need \textbf{12 \gameterm{rep} } to fill the \gameterm{rep}  tracker on your crew sheet. When you fill the tracker, do one of the following:

\begin{itemize}
	\item If your hold is weak, it becomes strong. \textbf{Reset your }\gameterm{rep} \textbf{ to zero}.
	\item If your hold is strong, you can pay to increase your crew Tier by one. This costs \gameterm{coin}  equal to your \textbf{new Tier x 8}. As long as your \gameterm{rep}  tracker is full, you don’t earn new \gameterm{rep}  (12 is the max). Once you pay and increase your Tier, \textbf{reset your }\gameterm{rep} \textbf{ to zero} and \textbf{reduce your hold to weak}.
\end{itemize}

\section{Turf}

Another way to contribute to the crew’s development is by acquiring \textbf{turf}. When you seize and hold territory, you establish a more stable basis for your \gameterm{rep} . Each piece of turf that you claim represents abstracted support for the crew (often a result of the fear you instill in the citizens on that turf).

Turf is marked on your \gameterm{rep}  tracker (see the example below). Each piece of turf you hold reduces the \gameterm{rep}  cost to develop by one. So, if you have 2 turf, you need 10 \gameterm{rep}  to develop. If you have 4 turf, you need 8 \gameterm{rep}  to develop. \textbf{You can hold a maximum of 6 turf.} When you develop and reset your \gameterm{rep} , \textbf{you keep the marks from all the turf you hold.}

@TODO table p37

Also, when you acquire turf, you expand the scope of your crew’s \textbf{hunting grounds}.

\subsection{Reducing hold}

You may perform an operation specifically to reduce the hold of another faction, if you know how they’re vulnerable. If the operation succeeds, the target faction loses 1 level of hold. If their hold is weak and it drops, the faction loses 1 Tier and stays weak.

When a faction is at war, it temporarily loses 1 hold.

Your crew can lose hold, too, following the same rules above. If your crew is Tier 0, with weak hold, and you lose hold for any reason, your lair comes under threat by your enemies or by a faction seeking to profit from your misfortune.

\section{Faction Status}

Your crew’s \textbf{status} with each faction indicates how well you are liked or hated. Status is rated from -3 to +3, with zero (neutral) being the default starting status. You track your status with each faction on the faction sheet.

When you create your crew, you assign some positive and negative status ratings to reflect recent history. The ratings will then change over time based on your actions in play.

\subsection{Faction status changes}

When you execute an operation, you gain -1 or -2 status with factions that are hurt by your actions. You may also gain +1 status with a faction that your operation helps. (If you keep your operation completely quiet then your status doesn’t change.) Your status may also change if you do a favor for a faction or if you refuse one of their demands.

\subsection{Faction status levels}

\begin{itemize}
	\item \textbf{+3: Allies}. This faction will help you even if it’s not in their best interest to do so. They expect you to do the same for them.
	\item \textbf{+2: Friendly.} This faction will help you if it doesn’t create serious problems for them. They expect you to do the same.
	\item \textbf{+1: Helpful.} This faction will help you if it causes no problems or significant cost for them. They expect the same from you.
	\item \textbf{0: Neutral}
	\item \textbf{-1: Interfering.} This faction will look for opportunities to cause trouble for you (or profit from your misfortune) as long as it causes no problems or significant cost for them. They expect the same from you.
	\item \textbf{-2: Hostile.} This faction will look for opportunities to hurt you as long as it doesn’t create serious problems for them. They expect you to do the same, and take precautions against you.
	\item \textbf{-3: War.} This faction will go out of its way to hurt you even if it’s not in their best interest to do so. They expect you to do the same, and take precautions against you. When you’re at war with any number of factions, your crew suffers +1 \gameterm{heat}  from scores, temporarily loses 1 hold, and PCs get only one downtime action rather than two. You can end a war by eliminating your enemy or by negotiating a mutual agreement to establish a new status rating.
\end{itemize}

\begin{quote}
	If your crew has weak hold when you go to war, the temporary loss of hold causes you to lose one Tier. When the war is over, restore your crew’s Tier back to its pre-war level.
\end{quote}

\section{Claims}

Each crew sheet has a map of claims available to be seized. The claim map displays a default roadmap for your crew type. Claims should usually be seized in an orderly sequence, by following the paths from the central square, the crew’s lair.

\emph{However, you may attempt to seize any claim on your map}, ignoring the paths (or even seek out a special claim not on your map) but these operations will always be especially difficult and require exceptional efforts to discover and achieve.

\subsection{Seizing a claim}

Every claim is already controlled by a faction. To acquire one for yourself, you have to take it from someone else. To seize a claim, tell the GM which claim on your map your crew intends to capture. The GM will detail the claim with a location and a description and will tell you which faction currently controls that claim. Or the GM might offer you a choice of a few options if they’re available.

If you choose to ignore the roadmap paths when seizing a claim, the GM might tell you that you’ll need to investigate and gather information in order to discover a claim of that type before you can attempt to seize it.

Execute the operation like any other \textbf{score}, and if you succeed, you seize the claim and the targeted faction loses the claim.

Seizing a claim is a serious attack on a faction, usually resulting in -2 faction status with the target, and potentially +1 status with its enemies.

As soon as you seize a claim, you enjoy the listed benefit for as long as you hold the claim. Some claims count as \textbf{turf}. Others provide special benefits to the crew, such as bonus dice in certain circumstances, extra \gameterm{coin}  generated for the crew’s treasury, or new opportunities for action.

\subsection{Losing a claim}

An enemy faction may try to seize a claim that your crew holds. You can fight to defend it, or negotiate a deal with the faction, depending on the situation. If you lose a claim, you lose all the benefits of that claim. If your lair is lost, you lose the benefits of all of your claims until you can restore your lair or establish a new one. To restore or establish a new lair, accomplish a score to do so.

\chapter{Advancement}

\section{PC Advancement}

Each player keeps track of the experience points (\textbf{xp}) that their character earns.

During the game session, mark xp:

\begin{itemize}
	\item When you make a \textbf{desperate action roll}. Mark 1 xp in the attribute for the action you rolled. For example, if you roll a desperate \gameterm{Skirmish}  action, you mark xp in \gameterm{Prowess} \emph{.} When you roll in a \textbf{group action} that’s desperate, you also mark xp.
\end{itemize}

At the end of the session, review the \textbf{xp triggers} on your character sheet. For each one, mark 1 xp if it happened at all, or mark 2 xp if it happened a lot during the session. The xp triggers are:

\begin{itemize}
	\item \textbf{Your playbook-specific xp trigger}. For example, the Cutter’s is \emph{“Address a challenge with violence or coercion.”} To “address a challenge,” your character should attempt to overcome a tough obstacle or threat. It doesn’t matter if the action is successful or not. You get xp either way.
	\item \textbf{You expressed your beliefs, drives, heritage, or background.} Your character’s beliefs and drives are yours to define, session to session. Feel free to tell the group about them when you mark xp.
	\item \textbf{You struggled with issues from your vice or traumas}. Mark xp for this if your vice tempted you to some bad action or if a trauma condition caused you trouble. Simply indulging your vice doesn’t count as struggling with it (unless you \textbf{overindulge}).
\end{itemize}

You may mark end-of-session xp on any xp tracks you want (any attribute or your playbook xp track).

When you fill an xp track, clear all the marks and take an \textbf{advance}. When you take an advance from your playbook track, you may choose an additional \textbf{special ability}. When you take an advance from an attribute, you may add an \textbf{additional action dot} to one of the actions under that attribute.

\begin{quote}
	Nadja is playing a Hound. At the end of the session, she reviews her xp triggers and tells the group how much xp she’s getting. She rolled two desperate \gameterm{Hunt}  actions during the session, so she marked 2 xp on her \gameterm{Insight}  xp track. She addressed several challenges with tracking or violence, so she marks 2 xp for that. She expressed her Iruvian heritage many times when dealing with the Red Sashes, so she takes 2 xp for that. She also showcased her character’s beliefs, but 2 xp is the maximum for that category, so she doesn’t get any more. She didn’t struggle with her vice or traumas, so no xp there. That’s 4 xp at the end of the session. She decides to put it all in her \gameterm{Insight}  xp track. This fills the track, so she adds a new action dot in \gameterm{Hunt} .
\end{quote}

You can also earn xp by \textbf{training} during downtime. When you train, mark xp in one of your attributes or in your playbook. A given xp track can be trained only once per downtime phase.

\section{Crew Advancement}

At the end of the session, review the crew xp triggers and mark 1 crew xp for each item that occurred during the session. If an item occurred multiple times or in a major way, mark 2 crew xp for it. The crew xp triggers are:

\begin{itemize}
	\item \textbf{Your crew-specific xp trigger}. For example, the Smugglers’ is \emph{“Execute a smuggling operation or acquire new clients or contraband sources.”} If the crew successfully completed an operation from this trigger, mark xp.
	\item \textbf{Contend with challenges above your current station.} If you tangled with higher Tiers or more dangerous opposition, mark xp for this.
	\item \textbf{Bolster your crew’s reputation or develop a new one.} Review your crew’s reputation. Did you do anything to promote it? Also mark xp if you developed a new reputation for the crew.
	\item \textbf{Express the goals, drives, inner conflict, or essential nature of the crew.} This one is very broad! Essentially, did anything happen that highlighted the specific elements that make your crew unique?
\end{itemize}

\textbf{When you fill your crew advancement tracker}, clear the marks and take a new \textbf{special ability }or mark \textbf{two crew upgrade boxes}.

\begin{quote}
	For example, when a crew of Assassins earns a crew advance, they could take a new special ability, like \textbf{Predators}. Or they could mark two upgrades, like \textbf{Ironhook Contacts} and \textbf{Resolve} \textbf{Training}.
\end{quote}

Say how you’ve obtained this new ability or upgrades for the crew. \emph{Where did it come from? How does it become a new part of the crew?}

\subsection{Profits}

Every time the crew advances, \textbf{each PC gets} \textbf{s}tash** equal to the crew Tier+2, to represent profits generated by the crew as they’ve been operating.

\part{The Characters}

\chapter{Characters}

Every player character is familiar with all of the various feats of skulduggery represented by the \textbf{actions} of the game. They’re all able to \gameterm{Skirmish}  in a knife-fight, \gameterm{Prowl}  in the shadows, \gameterm{Attune}  to strange energy, \gameterm{Consort}  with contacts for information, and so on.

Of course, you’ll also have your specializations and skills, the qualities that make your character uniquely effective. You might want the ability to compel obedience from ghosts and channel arcane energy through your body, or maybe you want to manipulate the network of the underworld to your advantage and see danger before it strikes, or maybe you just want to be the deadliest fighter with a blade. In this chapter, you’ll learn how to create your own unique scoundrel and choose the abilities that suit the style of play you prefer.

\chapter{Character creation}

\section{Create playbooks}

A playbook is what we call the sheet with all the specific rules to play a certain character type in a Blades-powered game. For example, you might create a playbook called a \textbf{Soldier}, with special abilities related to battle, or a playbook called a \textbf{Medic}, with special abilities related to field medicine.

When you choose a playbook, you’re choosing a set of \textbf{special abilities} (which give your character ways to break the rules in various ways) and a set of \textbf{xp triggers} (which determine how you earn experience for character advancement). But every playbook represents a scoundrel at heart. The Cutter has special abilities related to combat, but that doesn’t mean they’re “the fighter” of the game. Any character type can fight well. Think of your playbook as an area of focus and preference, but not a unique skill set.

This is why we call them “playbooks” rather than “character classes” or “archetypes.” You’re selecting the set of initial action ratings and special abilities that your character has access to---but you’re not defining their immutable essence or true nature. Your character will grow and change over time; who they become is part of the fun of playing the game.

Once you’ve chosen your playbook, follow the steps below to complete your character.

\section{Choose a heritage}

Your character’s \textbf{heritage} describes where their family line is from. When you choose a heritage, write a detail about your family life on the line above.

\section{Choose a background}

Your character’s \textbf{background} describes what they did before they joined the crew. Choose a background and then write a detail about it that’s specific to your character.

\section{Assign four action dots}

Your playbook begins with three action dots already placed. You get to add four more dots (so you’ll have seven total). At the start of the game, no action rating may have more than two dots (unless a special ability tells you otherwise). Assign your four dots like this:

\begin{itemize}
	\item Put one dot in any action that you feel reflects your character’s \textbf{heritage}.
	\item Put one dot in any action that you feel reflects your character’s \textbf{background}.
	\item Assign two more dots anywhere you please (max rating is 2, remember).
\end{itemize}

\section{Choose a special ability}

Take a look at the special abilities for your playbook and choose one. If you can’t decide which one to pick, go with the first one on the list---it’s placed there as a good default choice.

\subsection{Special Armor}

Some special abilities refer to your \textbf{special armor}. Each character sheet has a set of three boxes to track usage of armor (standard, heavy, and special). If you have any abilities that use your special armor, tick its box when you activate one of them. If you don’t have any special abilities that use special armor, then you can’t use that armor box at all.

\section{Choose one close friend and one rival}

Each playbook has a list of NPCs that your character knows. Choose one from the list who is a close relationship (a good friend, a lover, a family relation, etc.). Mark the upward-pointing triangle next to their name. Then choose another NPC on the list who’s your rival or enemy. Mark the downward-pointing triangle next to their name.

\section{Choose your vice}

Every character is in thrall to some vice or another, which they indulge to deal with stress. Choose a vice from the list, and describe it on the line above with the specific details and the name and location of your \textbf{vice purveyor}.

\begin{itemize}
	\item \gameterm{Faith: } You’re dedicated to an unseen power, forgotten god, ancestor, etc.
	\item \gameterm{Gambling:}  You crave games of chance, betting on sporting events, etc.
	\item \gameterm{Luxury: } Expensive or ostentatious displays of opulence.
	\item \gameterm{Obligation: } You’re devoted to a family, a cause, an organization, a charity, etc.
	\item \gameterm{Pleasure:}  Gratification from lovers, food, drink, drugs, art, theater, etc.
	\item \gameterm{Stupor: } You seek oblivion in the abuse of drugs, drinking to excess, getting beaten to a pulp in the fighting pits, etc.
	\item \gameterm{Weird: } You experiment with strange essences, consort with rogue spirits, observe bizarre rituals or taboos, etc.
\end{itemize}

\section{Record your name, alias, \& look}

Choose a name for your character from the sample list, or create your own. If your character uses an alias or nickname in the underworld, make a note of it. Record a few evocative words that describe your character’s look (samples provided on the next page).

\section{Review your details}

Take a look at the details on your character sheet, especially the \textbf{experience triggers} for your playbook (like “Earn xp when you address a challenge with knowledge or arcane power,” for example) and the \textbf{special items} available to a character of your type (like the Whisper’s spirit mask, for example). You begin with access to all of the items on your sheet, so don’t worry about picking specific things---you’ll decide what your character is carrying later on, when you’re on the job.

That’s it! Your character is ready for play. When you start the first session, the GM will ask you some questions about who you are, your outlook, or some past events. If you don’t know the answers, make some up. Or ask the other players for ideas.

\section{Character creation summary}

@TODO style dropcaps.

\begin{enumerate}
	\item \textbf{Choose a playbook.} Your playbook represents your character’s reputation, their special abilities, and how they advance.
	\item \textbf{Choose a heritage.} Detail your choice with a note about your family life. \emph{For example, Ore miners, now war refugees.}
	\item \textbf{Choose a background.} Detail your choice with your specific history. \emph{For example, Labor: Hunter, mutineer.}
	\item \textbf{Assign four action dots.} No action may begin with a rating higher than 2 during character creation. \emph{(After creation, action ratings may advance up to 3. When you unlock the Mastery advance for your crew, you can advance actions up to rating 4.)}
	\item \textbf{Choose a special ability.} They’re in the gray column in the middle of the character sheet. If you can’t decide, choose the first ability on the list. It’s placed there as a good first option.
	\item \textbf{Choose a close friend and a rival.} Mark the one who is a close friend, long-time ally, family relation, or lover (the upward-pointing triangle). Mark one who is a rival, enemy, scorned lover, betrayed partner, etc. (the downward-pointing triangle).
	\item \textbf{Choose your vice.} Pick your preferred type of vice, detail it with a short description and indicate the name and location of your vice purveyor.
	\item \textbf{Record your name, alias, and look.} Choose a name, an alias (if you use one), and jot down a few words to describe your look. Examples are provided on the preceding page.
\end{enumerate}

\section{Loadout}

You have access to all of the \textbf{items} on your character sheet. For each operation, decide what your character’s \textbf{load} will be. During the operation, you may say that your character has an item on hand by checking the box for the item you want to use---up to a number of items equal to your chosen load. Your load also determines your movement speed and conspicuousness:

\begin{itemize}
	\item \gameterm{1-3 load:} \textbf{Light}. \emph{You’re faster, less conspicuous; you blend in with citizens.}
	\item \gameterm{4/5 load: } \textbf{Normal}. \emph{You look like a scoundrel, ready for trouble.}
	\item \gameterm{6 load: } \textbf{Heavy.} \emph{You’re slower. You look like an operative on a mission.}
	\item \gameterm{7-9 load: } \textbf{Encumbered.} \emph{You’re overburdened and can’t do anything except move very slowly.}
\end{itemize}

Some special abilities (like the Cutter’s \gameterm{Mule}  ability or Assassin’s Rigging) increase the load limits.

Some items count as two items for load (they have two connected boxes). \emph{Items in italics don’t count toward your load.}

You don’t need to select specific items now.  Review your personal items and the standard item descriptions.

\chapter{Character playbook}

\section{Short Descriptor of Character}

Medium length description of the character here. Include what kind of activities they normally partake in.

\textbf{Include the XP triggers for the characters here.} At the end of a session mark XP if you addressed a challenge with: insert a short list of methods or actions here. Examples could include: violence, coersion, knowledge, charm, audacity, calculation, deception, influence, stealth, evasion, technical skill, mayham, tracking, or occult powers.

@TODO tables 51


\section{Starting builds}

\emph{If you want some guidance when you assign your four starting action dots and special ability, use one of these templates.}

\textbf{Example 1.} 4 Action Dots assigned plus a special ability

\textbf{Example 2.} 4 Action Dots assigned plus a special ability

\textbf{Example 3.} 4 Action Dots assigned plus a special ability

\textbf{Example 4.} 4 Action Dots assigned plus a special ability

\section{Friends, rivals}

A list of five possible friends or rivals go here, along with descriptions of each. Some possibilities include: A spy, a bounty hunte, a pugalist, a cold killer, an extortionist, a physicker, an assassin, a sentinel, an apothocary, a priestess, a noble, a city clerk, an officer, an inspector, a beggar, a locksmith, a gang leader, a drug dealer, a tavern owner, a porstitue, a jail-bird, an information broker, a servant, an archivist, or a supernatural entitity.

Questions can inlclue how you know the friend, what they’ve done for you,  what you do for them, and what kind of relationship you have.

\section{Special abilities}

Choose seven special abilities for you playbook. Here are several examples.

\subsection{Battleborn}

You may expend your \textbf{special armor} to reduce harm from an attack in combat or to \textbf{push yourself} during a fight.

\begin{quote}
	When you use this ability, tick the special armor box on your playbook sheet. If you “reduce harm” that means the level of harm you’re facing right now is reduced by one. If you use this ability to push yourself, you get one of the benefits (+1d, +1 effect, act despite severe harm) but you don’t take 2 stress. Your special armor is restored at the beginning of downtime.
\end{quote}

\subsection{Bodyguard}

When you \textbf{protect} a teammate, take \textbf{+1d} to your resistance roll. When you gather info to anticipate possible threats in the current situation, you get \textbf{+1 effect}.

\begin{quote}
	The protect teamwork maneuver lets you face a consequence for a teammate. If you choose to resist that consequence, this ability gives you +1d to your resistance roll. Also, when you read a situation to gather information about hidden dangers or potential attackers, you get +1 effect---which means more detailed information.
\end{quote}

\subsection{Arcane fighter}

You may imbue your hands, melee weapons, or tools with spirit energy. You gain \textbf{potency} in combat vs. the supernatural..

\begin{quote}
	When you imbue yourself with arcane energy, how do you do it? What does it look like when the energy manifests?
\end{quote} 

\subsection{Leader}

When you \gameterm{Command}  a \textbf{cohort} in combat, they continue to fight when they would otherwise \textbf{break }(they’re not taken out when they suffer level 3 harm). They gain \textbf{+1 effect} and \textbf{1 armor}.

\begin{quote}
	This ability makes your cohorts more effective in battle and also allows them to resist harm by using armor. While you lead your cohorts, they won’t stop fighting until they take fatal harm (level 4) or you order them to cease. What do you do to inspire such bravery in battle?
\end{quote} 

\subsection{Mule}

Your load limits are higher. Light: 5. Normal: 7. Heavy: 8.

\begin{quote}
	This ability is great if you want to wear heavy armor and pack a heavy weapon without attracting lots of attention. Since your exact gear is determined on-the-fly during an operation, having more load also gives you more options to get creative with when dealing with problems during a score.
\end{quote} 

\subsection{Not to be trifled with}

You can \textbf{push yourself} to do one of the following: \emph{perform a feat of physical force that verges on the superhuman---engage a small gang on equal footing in close combat.}

\begin{quote}
	When you push yourself to activate this ability, you still get one of the normal benefits of pushing yourself (+1d, +1 effect, etc.) in addition to the special ability.
\end{quote} 

\begin{quote}
	If you perform a feat that verges on the superhuman, you might break a metal weapon with your bare hands, tackle a galloping horse, lift a huge weight, etc. If you engage a small gang on equal footing, you don’t suffer reduced effect due to scale against a small gang (up to six people).
\end{quote} 

\subsection{Savage}

When you unleash physical violence, it’s especially frightening. When you \gameterm{Command}  a frightened target, take \textbf{+1d}.

\begin{quote}
	You instill fear in those around you when you get violent. How they react depends on the person. Some people will flee from you, some will be impressed, some will get violent in return. The GM judges the response of a given NPC.
\end{quote} 

\begin{quote}
	In addition, when you \gameterm{Command}  someone who’s affected by fear (from this ability or otherwise), take +1d to your roll.
\end{quote} 

\subsection{Vigorous}

You recover from harm faster. Permanently fill in one of your healing clock segments. Take \textbf{+1d} to healing treatment rolls.

\begin{quote}
	Your healing clock becomes a 3-clock, and you get a bonus die when you recover.
\end{quote} 

\subsection{Sharpshooter}

You can \textbf{push yourself} to do one of the following: \emph{make a ranged attack at extreme distance beyond what’s normal for the weapon---unleash a barrage of rapid fire to suppress the enemy.}

\begin{quote}
	When you push yourself to activate this ability, you still get one of the normal benefits of pushing yourself (+1d, +1 effect, etc.) in addition to the special ability.
\end{quote} 

\begin{quote}
	The first use of this ability allows you to attempt long-range sniper shots that would otherwise be impossible with  typical rudimentary firearms. The second use allows you to keep up a steady rate of fire in a battle (enough to “suppress” a small gang up to six people), rather than stopping for a slow reload or discarding a gun after each shot. When an enemy is suppressed, they’re reluctant to maneuver or attack (usually calling for a fortune roll to see if they can manage it).
\end{quote} 

\subsection{Focused}

You may expend your \textbf{special armor} to resist a consequence of surprise or mental harm (fear, confusion, losing track of someone) or to \textbf{push yourself} for ranged combat or tracking.

\begin{quote}
	When you use this ability, tick the special armor box on your playbook sheet. If you “resist a consequence” of the appropriate type, you avoid it completely. If you use this ability to push yourself, you get one of the benefits (+1d, +1 effect, act despite severe harm) but you don’t take 2 stress. Your special armor is restored at the beginning of downtime.
\end{quote} 

\subsection{Scout}

When you gather information to discover the location of a target, you get \textbf{+1 effect}. When you hide in a prepared position or use camouflage you get \textbf{+1d} to rolls to avoid detection.

\begin{quote}
	A “target” can be a person, a destination, a good ambush spot, an item, etc.
\end{quote} 

\subsection{Tough as nails}

Penalties from harm are one level less severe (though level 4 harm is still fatal).

\begin{quote}
	With this ability, level 3 harm doesn’t incapacitate you; instead you take -1d to your rolls (as if it were level 2 harm). Level 2 harm affects you as if it were level 1 (less effect). Level 1 harm has no effect on you (but you still write it on your sheet, and must recover to heal it). Record the harm at its original level---for healing purposes, the original harm level applies.
\end{quote} 

\subsection{Vengeful}

You gain an additional \textbf{xp trigger}: \emph{You got payback against someone who harmed you or someone you care about.} If your crew helped you get payback, also mark crew xp.

\subsection{Alchemist}

When you \textbf{invent} or \textbf{craft} a creation with \emph{alchemical} features, you get \textbf{+1 result level} to your roll (a \gameterm{1-3}  becomes a \gameterm{4/5} , etc.). You begin with one special formula already known.

\begin{quote}
	Follow the Inventing procedure with the GM to define your first special alchemical formula.
\end{quote} 

\subsection{Analyst}

During downtime, you get \textbf{two ticks} to distribute among any long term project clocks that involve investigation or learning a new formula or design plan.

\subsection{Artificer}

When you \textbf{invent} or \textbf{craft} a creation with \emph{spark-craft} features, you get \textbf{+1 result level} to your roll (a \gameterm{1-3}  becomes a \gameterm{4/5} , etc.). You begin with one special design already known.

\begin{quote}
	Follow the Inventing procedure with the GM to define your first spark-craft design.
\end{quote} 

\section{Fortitude}

You may expend your \textbf{special armor} to resist a consequence of fatigue, weakness, or chemical effects, or to \textbf{push yourself} when working with technical skill or handling alchemicals.

\begin{quote}
	When you use this ability, tick the special armor box on your playbook sheet. If you “resist a consequence” of the appropriate type, you avoid it completely. If you use this ability to push yourself, you get one of the benefits (+1d, +1 effect, act despite severe harm) but you don’t take 2 stress. Your special armor is restored at the beginning of downtime.
\end{quote} 

\subsection{Supernatural ward}

When you \gameterm{Wreck}  an area with arcane substances, ruining it for any other use, it becomes anathema or enticing to the supernatural (your choice).

\begin{quote}
	If you make an area anathema to the supernatural, they will do everything they can to avoid it, and will suffer torment if forced inside the area. If you make an area enticing to spirits, they will seek it out and linger in the area, and will suffer torment if forced to leave. This effect lasts for several days over an area the size of a small room. Particularly powerful or prepared entities may roll their quality or arcane magnitude to see how well they’re able to resist the effect.
\end{quote} 

\subsection{Saboteur}

When you \gameterm{Wreck} , your work is much quieter than it should be and the damage is very well-hidden from casual inspection.

\begin{quote}
	You can drill holes in things, melt stuff with acid, even use a muffled explosive, and it will all be very quiet and extremely hard to notice.
\end{quote} 

\subsection{Infiltrator}

You are not affected by \textbf{quality} or \textbf{Tier} when you bypass security measures.

\begin{quote}
	This ability lets you contend with higher-Tier enemies on equal footing. When you’re cracking a safe, picking a lock, or sneaking past elite guards, your effect level is never reduced due to superior Tier or quality level of your opposition.
\end{quote} 

\begin{quote}
	Are you a renowned safe cracker? Do people tell stories of how you slipped under the noses of two Chief Inspectors, or are your exceptional talents yet to be discovered?
\end{quote} 

\subsection{Ambush}

When you attack from hiding or spring a trap, you get \textbf{+1d} to your roll.

\begin{quote}
	This ability benefits from preparation--- so don’t forget you can do that in a flashback.
\end{quote} 

\subsection{Daredevil}

When you roll a \textbf{desperate} action, you get \textbf{+1d} to your roll if you also take \textbf{-1d} to any resistance rolls against consequences from your action.

\begin{quote}
	This special ability is a bit of a gamble. The bonus die helps you, but if you suffer consequences, they’ll probably be more costly to resist. But hey, you’re a daredevil, so no big deal, right?
\end{quote} 

\subsection{The devil’s footsteps}

You can \textbf{push yourself} to do one of the following: \emph{perform a feat of athletics that verges on the superhuman---maneuver to confuse your enemies so they mistakenly attack each other.}

\begin{quote}
	When you push yourself to activate this ability, you still get one of the normal benefits of pushing yourself (+1d, +1 effect, etc.) if you’re making a roll, in addition to the special ability.
\end{quote} 

\begin{quote}
	If you perform an athletic feat (running, tumbling, balance, climbing, etc.) that verges on the superhuman, you might climb a sheer surface that lacks good hand-holds, tumble safely out of a three-story fall, leap a shocking distance, etc.
\end{quote} 

\begin{quote}
	If you maneuver to confuse your enemies, they attack each other for a moment before they realize their mistake. The GM might make a fortune roll to see how badly they harm or interfere with each other.
\end{quote} 

\subsection{Expertise}

Choose one of your action ratings. When you lead a group action using that action, you can suffer only 1 stress at most, regardless of the number of failed rolls.

\begin{quote}
	This special ability is good for covering for your team. If they’re all terrible at your favored action, you don’t have to worry about suffering a lot of stress when you lead their group action.
\end{quote} 

\subsection{Reflexes}

When there’s a question about who acts first, the answer is you.

\begin{quote}
	This ability gives you the initiative in most situations. Some specially trained NPCs might also have reflexes, but otherwise, you’re always the first to act, and can interrupt anyone else who tries to beat you to the punch. This ability usually doesn’t negate the need to make an action roll that you would otherwise have to make, but it may improve your position or effect.
\end{quote} 

\subsection{Shadow}

You may expend your \textbf{special armor} to resist a consequence from detection or security measures, or to \textbf{push yourself} for a feat of athletics or stealth.

\begin{quote}
	When you use this ability, tick the special armor box on your playbook sheet. If you “resist a consequence” of the appropriate type, you avoid it completely. If you use this ability to push yourself, you get one of the benefits (+1d, +1 effect, act despite severe harm) but you don’t take 2 stress. Your special armor is restored at the beginning of downtime.
\end{quote} 

\subsection{Rook’s gambit}

Take \textbf{2 stress} to roll your best action rating while performing a different action. Say how you adapt your skill to this use.

\begin{quote}
	This is the “jack-of-all-trades” ability. If you want to attempt lots of different sorts of actions and still have a good dice pool to roll, this is the special ability for you.
\end{quote} 

\subsection{Like looking into a mirror}

You can always tell when someone is lying to you.

\begin{quote}
	This ability works in all situations without restriction. It is very powerful, but also a bit of a curse. You see though every lie, even the kind ones.
\end{quote} 

\subsection{A little something on the side}

At the end of each downtime phase, you earn \textbf{+2 stash}.

\begin{quote}
	Since this money comes at the end of downtime, after all downtime actions are resolved, you can’t remove it from your stash and spend it on extra activities until your next downtime phase.
\end{quote} 

\subsection{Mesmerism}

When you \gameterm{Sway}  someone, you may cause them to forget that it’s happened until they next interact with you.

\begin{quote}
	The victim’s memory “glosses over” the missing time, so it’s not suspicious that they’ve forgotten something. When you next interact with the victim, they remember everything clearly, including the strange effect of this ability.
\end{quote} 

\subsection{Subterfuge}

You may expend your \textbf{special armor} to resist a consequence from suspicion or persuasion, or to \textbf{push yourself} for subterfuge.

\begin{quote}
	When you use this ability, tick the special armor box on your playbook sheet. If you “resist a consequence” of the appropriate type, you avoid it completely. If you use this ability to push yourself, you get one of the benefits (+1d, +1 effect, act despite severe harm) but you don’t take 2 stress. Your special armor is restored at the beginning of downtime.
\end{quote} 

\subsection{Trust in me}

You get \textbf{+1d} vs. a target with whom you have an intimate relationship.

\begin{quote}
	This ability isn’t just for social interactions. Any action can get the bonus. “Intimate” is for you and the group to define, it need not exclusively mean romantic intimacy.
\end{quote} 

\subsection{Foresight}

Two times per score you can \textbf{assist} a teammate without paying stress. Describe how you prepared for this.

\begin{quote}
	You can narrate an event in the past that helps your teammate now, or you might explain how you expected this situation and planned a helpful contingency that you reveal now.
\end{quote} 

\subsection{Calculating}

Due to your careful planning, during downtime, you may give yourself or another crew member \textbf{+1 downtime activity}.

\begin{quote}
	If you forget to use this ability during downtime, you can still activate it during the score and flashback to the previous downtime when the extra activity happened.
\end{quote} 

\subsection{Connected}

During downtime, you get \textbf{+1 result level} when you \textbf{acquire an asset} or \textbf{reduce heat}.

\begin{quote}
	Your array of underworld connections can be leveraged to loan assets, pressure a vendor to give you a better deal, intimidate witnesses, etc.
\end{quote} 

\subsection{Mastermind}

You may expend your \textbf{special armor} to \textbf{protect} a teammate, or to \textbf{push yourself} when you gather information or work on a long-term project.

\begin{quote}
	When you use this ability, tick the special armor box on your playbook sheet. If you protect a teammate, this ability negates or reduces the severity of a consequence or harm that your teammate is facing. You don’t have to be present to use this ability---say how you prepared for this situation in the past. If you use this ability to push yourself, you get one of the benefits (+1d, +1 effect, act despite severe harm) but you don’t take 2 stress. Your special armor is restored at the beginning of downtime.
\end{quote} 

\subsection{Weaving the web}

You gain \textbf{+1d} to \gameterm{Consort}  when you gather information on a target for a score. You get \textbf{+1d} to the \textbf{engagement roll} for that operation.

\begin{quote}
	Your network of underworld connections can always be leveraged to gain insight for a job---even when your contacts aren’t aware that they’re helping you.
\end{quote} 

\subsection{Arcane mind}

You’re always aware of supernatural entities in your presence. Take \textbf{+1d} whenever you \textbf{gather information} about the supernatural by any means.

\subsection{Iron will}

You are immune to the terror that some supernatural entities inflict on sight. When you make a \textbf{resistance roll }with \gameterm{Resolve} , take \textbf{+1d}.

\begin{quote}
	With this ability, you do not freeze up or flee when confronted by any kind of supernatural entity or strange occult event.
\end{quote} 

\subsection{Ritual}

You know the arcane methods to perform ritual sorcery. You can \gameterm{Study}  an occult ritual (or create a new one) to summon a supernatural effect or being. You begin with one ritual already learned.

\begin{quote}
	Without this special ability, the study and practice of rituals leaves you utterly vulnerable to the powers you supplicate. Such endeavors are not recommended.
\end{quote} 

\subsection{Strange methods}

When you \textbf{invent} or \textbf{craft} a creation with \emph{arcane} features, get \textbf{+1 result level} to your roll (a \gameterm{1-3}  becomes a \gameterm{4/5} , etc.). You begin with one arcane design already known.

\begin{quote}
	Follow the Inventing procedure with the GM  to define your first arcane design.
\end{quote} 

\subsection{Warded}

You may expend your \textbf{special armor} to resist a supernatural consequence, or to \textbf{push yourself} when you contend with or employ arcane forces.

\begin{quote}
	When you use this ability, tick the special armor box on your playbook sheet. If you resist a consequence, this ability negates or reduces its severity. If you use this ability to push yourself, you get one of the benefits (+1d, +1 effect, act despite severe harm) but you don’t take 2 stress. Your special armor is restored at the beginning of downtime.
\end{quote} 

\section{Character items}

Add a list of five distinctive items that are either Fine (+1 quality), light (reduced load), rare, or otherwise specific to this character.

Some examples include: Weapons, clothes, arcane impliments, specialized tools, custom gear, capable pets, or supernatural objects.

\part{The Crew}

\chapter{Crew creation}

\section{Choose a crew type}

Each Blades-powered game should have crew types that suit its setting and premise. These might be organizations, gangs, ships, strongholds, etc. Three to six options are good.

Your crew type determines the scores that you’ll focus on, as well as a selection of special abilities that support that kind of action. The crew type isn’t meant to be restrictive---a crew of smugglers might sometimes engage in extortion  or sell contraband ---but the core activity of the crew type is the most frequent way they earn \gameterm{coin}  and xp for advancement.

Like a character playbook, your crew type is also how you’re known in the underworld. The criminal factions and institutions think of you as “assassins” or “a cult” etc., and will treat you accordingly.

Your crew begins with 2 \gameterm{coin}  in its coffers (the remains of the PCs’ savings). You are \textbf{Tier 0}, with \textbf{strong hold} and \textbf{0} \gameterm{rep} .

\section{Choose an initial reputation \& lair}

Your crew has just formed and acquired a lair. Given this group of characters and their previous escapades, what initial \textbf{reputation} would you have among the factions of the underworld? Choose one from the list at right (or create your own).

\begin{itemize}
	\item Ambitious
	\item Brutal
	\item Daring
	\item Honorable
	\item Professional
	\item Savvy
	\item Subtle
	\item Strange
\end{itemize}

You earn xp when you bolster your crew’s reputation, so think of this as another cue to indicate what sorts of action you want in the game. Will you be recklessly ambitious, targeting higher-Tier targets? Will you take on daring scores that others deem too risky? Are you interested in the strange and the weird?

Also, talk about where the crew makes its \textbf{lair}. You begin at Tier 0, so it’s probably a very modest or abandoned sort of place.

\section{Establish your hunting grounds}

Your crew is brand-new, but you have chosen some small part of a district as your \textbf{hunting grounds}. This is the area that you usually target for your scores, and you know it well. Your hunting grounds don’t have to be in the same district as your lair. The area is small, only three or four city blocks---but it’s still an intrusion on someone. The entire city is divided among larger, stronger factions. The GM will tell you which faction claims the area, then you decide how to deal with them:

\begin{itemize}
	\item Pay them off. Give them 1 \gameterm{coin}  in exchange for giving you room to work.
	\item Pay the faction 2 \gameterm{coin}  as a show of respect and gain \textbf{+1 status }with them.
	\item Keep your money and take \textbf{-1 status} with that faction.
\end{itemize}

Your hunting grounds are useful for a particular type of criminal operation. Each crew type has a list of different operation types for their hunting grounds. For instance, Assassins have \emph{Accident}, \emph{Disappearance}, \emph{Murder}, or \emph{Ransom} as options. Choose one of these operation types as your preference.

When you prepare to execute an operation of your preferred type on your hunting grounds, you  get \textbf{+1d} to any \textbf{gather information} rolls and a free additional \textbf{downtime activity} to contribute to that operation. This can help you discover an opportunity, acquire an asset you might need for the job, find an appropriate client, etc.

When you acquire \textbf{turf} you also expand the size and/or type of your hunting grounds. Detail the new area and/or methods with help from the GM.

\section{Choose a special ability}

Take a look at the special abilities for your crew and choose one. If you can’t decide which one to pick, go with the first one on the list---it’s placed there as a good default choice. It’s important to pick a special ability that everyone is excited about. You can get more special abilities in the future by earning xp.

Just like picking the crew type, reputation, lair, and hunting grounds, choosing a special ability is another chance to focus the game down to a more specific range of possibilities. That’s a lot to work with, and it helps get the game going in a strong direction from the very beginning.

\section{Assign crew upgrades}

An upgrade is a valuable asset that helps the crew in some way, like a boat or a gang (see the complete descriptions on the following page). Each crew type has two pre-selected \textbf{upgrades} that suit that crew (like \textbf{Prowess} \textbf{Training} and a\textbf{ gang} of Thugs for the Bravos crew).

You get to add \textbf{two additional upgrades} to your new crew (so you’ll have a total of four upgrades when you start). You can choose from the specific upgrades available to your crew type or the general upgrades on the crew sheet.

When you assign your two upgrades, the GM will tell you about two factions that are impacted by your choices:

\begin{itemize}
	\item One faction helped you get an upgrade. They like you, and you get \textbf{+1 status} with them. At your option, spend 1 \gameterm{coin } to repay their kindness, and take \textbf{+2 status} with them instead.
	\item One faction was screwed over when you got an upgrade. They don’t like you, and you get \textbf{-2 status} with them. At your option, spend 1 \gameterm{coin } to mollify them, and take \textbf{-1 status} with them instead.
\end{itemize}

You’ll be able get more upgrades in the future by earning xp

\section{Choose a favorite contact}

Take a look at your list of potential contacts on the crew sheet. Choose one contact who is a close friend, long-time ally, or partner in crime. The GM will tell you about two factions that are impacted by your choice:

\begin{itemize}
	\item One faction is also friendly with this contact, and you get \textbf{+1 status} with them.
	\item One faction is unfriendly with this contact, and you get \textbf{-1 status} with them.
\end{itemize}

At your option, these factions are even more concerned with this contact and so you take \textbf{+2} and \textbf{-2 status} instead.

\section{Crew upgrade Examples}

\begin{itemize}
	\item \gameterm{Boat house:} You have a boat, a dock on a waterway, and a small shack to store boating supplies. A second upgrade improves the boat with armor and more cargo capacity.
	\item \gameterm{Carriage House:}  You have a carriage, and draft animals to pull it, and a stable. A second upgrade improves the carriage with armor and larger, swifter steeds.
	\item \gameterm{Cohort:}  A cohort is a gang or a single expert NPC who works for your crew. For all the details on cohorts, see the following pages.
	\item \gameterm{Hidden Lair:}  Your lair has a secret location and is disguised to hide it from view. If your lair is discovered, use two downtime activities and pay \gameterm{coin}  equal to your Tier to relocate it and hide it once again.
	\item \gameterm{Mastery: } Your crew has access to master level training. You may advance your PCs’ action ratings to 4 (until you unlock this upgrade, PC action ratings are capped at 3). This costs four upgrade boxes to unlock.
	\item \gameterm{Quality:} \textbf{ }Each upgrade improves the quality rating\textbf{ of all the PCs’ items of that type, beyond the quality established by the crew’s Tier and fine items. You can improve the quality of }Documents\textbf{, }Gear\textbf{ (covers Burglary Gear and Climbing Gear), Arcane }Implements\textbf{, Subterfuge }Supplies\textbf{, }Tools\textbf{ (covers Demolitions Tools and Tinkering Tools), and }Weapons**. \emph{So, if you are Tier 0, with fine lockpicks (+1) and the Quality upgrade for gear (+1), you could contend equally with a Tier II quality lock.}
	\item \gameterm{Quarters:}  Your lair includes living quarters for the crew. Without this upgrade, each PC sleeps elsewhere, and is vulnerable when they do so.
	\item \gameterm{Secure Lair:} Your lair has locks, alarms, and traps to thwart intruders. A second upgrade improves the defenses to include arcane measures that work against spirits. \emph{You might roll your crew’s Tier if these measures are ever put to the test, to see how well they thwart an intruder.}
	\item \gameterm{Training: } If you have a Training upgrade, you earn 2 xp (instead of 1) when you train a given xp track during downtime (\gameterm{Insight} , \gameterm{Prowess} , \gameterm{Resolve} , or Playbook xp). This upgrade essentially helps you advance more quickly. See \textbf{Advancement}, page 40. \emph{If you have Insight Training, when you train Insight during downtime, you mark 2 xp on the \gameterm{Insight}  track (instead of just 1). If you have Playbook Training, you mark 2 xp on your playbook xp track when you train.}
	\item \gameterm{Vault:} Your lair has a secure vault, increasing your storage capacity for \gameterm{coin}  to 8. A second upgrade increases your capacity to 16.  A separate part of your vault can be used as a holding cell.
	\item \gameterm{Workshop:}  Your lair has a workshop appointed with tools for tinkering and alchemy, as well as a small library of books, documents, and maps. You may accomplish long-term projects with these assets without leaving your lair.
\end{itemize}

\section{Cohorts}

A \textbf{cohort} is a \textbf{gang} or an \textbf{expert} who works for your crew. To recruit a new cohort, spend\textbf{ two upgrades} and create them using the process below.

\subsection{Creating a gang}

Choose a \textbf{gang type} from the list below:

\begin{itemize}
	\item \gameterm{Adepts:} Scholars, tinkerers, occultists, and chemists.
	\item \gameterm{Rooks:}  Con artists, spies, and socialites.
	\item \gameterm{Rovers:} Sailors, carriage drivers, and deathlands scavengers.
	\item \gameterm{Skulks:}  Scouts, infiltrators, and thieves.
	\item \gameterm{Thugs:}  Killers, brawlers, and roustabouts.
\end{itemize}

A gang has \textbf{scale} and \textbf{quality} equal to your current crew Tier. It increases in scale and quality when your crew moves up in Tier.

\begin{quote}
	If your crew is Tier 0, your gang is quality 0 and scale 0 (1 or 2 people). When your crew is Tier II, your gang is quality 2 and scale 2 (12 people).
\end{quote} 

Some crew upgrades will add the “Elite” feature to a gang, which gives them +1d when they roll for a given Type. \emph{So, if you’re Tier I and have a gang of Elite Thugs (+1d), they would roll 2d when they try to kill a target.}

\subsection{Creating an expert}

Record the expert’s \textbf{type} (their specific area of expertise). They might be a \emph{Doctor}, an \emph{Investigator}, an \emph{Occultist}, an \emph{Assassin}, a \emph{Spy}, etc.

An expert has \textbf{quality} equal to your current crew Tier +1. Their scale is always zero (1 person). Your experts increase in quality when your crew moves up in Tier.

\subsection{Edges \& flaws}

When you create a cohort, give them one or two \textbf{edges} and an equal number of \textbf{flaws}.

\textbf{Edges}

\begin{itemize}
	\item \textbf{Fearsome: }The cohort is terrifying in aspect and reputation.
	\item \textbf{Independent:} The cohort can be trusted to make good decisions and act on their own initiative in the absence of direct orders.
	\item \textbf{Loyal:} The cohort can’t be bribed or turned against you.
	\item \textbf{Tenacious:} The cohort won’t be deterred from a task.
\end{itemize}

\textbf{Flaws}

\begin{itemize}
	\item \textbf{Principled:} The cohort has an ethic or values that it won’t betray.
	\item \textbf{Savage:} The cohort is excessively violent and cruel.
	\item \textbf{Unreliable:} The cohort isn’t always available, due to other obligations, stupefaction from their vices, etc.
	\item \textbf{Wild:} The cohort is drunken, debauched, and loud-mouthed.
\end{itemize}

\subsection{Modifying a cohort}

You can add an \textbf{additional type} to a gang or expert by spending two crew upgrades. When a cohort performs actions for which its types apply, it uses its full quality rating. Otherwise, its quality is zero. A given cohort can have up to two types.

\subsection{Using a cohort}

When you send a cohort to achieve a goal, roll their \textbf{quality} to see how it goes. Or, a PC can oversee the maneuver by leading a \textbf{group action}. If you direct the cohort with orders, roll \gameterm{Command} . If you participate in the action alongside the cohort, roll the appropriate action. The quality of any opposition relative to the cohort’s quality affects the position and effect of the action.

\begin{quote}
	The PCs crew of Hawkers want to run a rival gang out of the alley where they’re selling drugs. They send their gang of Thugs to go kick the interlopers out. The GM rolls 2d for the Thugs’ quality, and gets a \gameterm{3} . An hour later, the Thugs come back, beaten and bloody. One of them looks sheepish, “Those guys are tough, boss.” (The GM inflicts harm on the cohort, and they failed their goal.)
\end{quote} 

\begin{quote}
	The next day, the crew boss goes back and leads a group action, rolling her 3d in \gameterm{Skirmish}  alongside the Thugs’ 2d. The boss gets a \gameterm{6}  this time---they beat the tar out of the other gang and send them packing (at least for now).
\end{quote} 

\section{Cohort harm \& Healing}

Cohorts suffer harm similarly to PCs. A cohort can suffer four levels of harm:
\begin{enumerate}
	\item \textbf{Weakened.} The cohort has reduced effect.
	\item \textbf{Impaired.} The cohort operates with reduced quality (-1d).
	\item \textbf{Broken.} The cohort can’t do anything until they recover.
	\item \textbf{Dead.} The cohort is destroyed.
\end{enumerate}

All of your cohorts heal during downtime. If circumstances are amenable for recovery, each cohort removes one level of harm (or two levels of harm instead, if a PC spends a downtime activity helping them recuperate).

If a cohort is destroyed, it may be replaced. Spend \gameterm{coin}  equal to your Tier +2 to restore it, plus two downtime activities to recruit new gang members, or hire a new expert.

\section{Crew creation summary}

@TODO dropcaps

\begin{enumerate}
	\item \textbf{Choose a crew type.} The crew type determines the group’s purpose, their special abilities, and how they advance.
		You begin at \textbf{Tier 0}, with \textbf{strong hold} and 0 \gameterm{rep} . You start with 2 \gameterm{coin} .
	\item \textbf{Choose an initial reputation and lair.} Choose how other underworld factions see you:\emph{Ambitious---Brutal---Daring---Honorable---Professional---Savvy---Subtle---Strange.} Look at the map and pick a district in which to place your lair. Describe the lair.
	\item \textbf{Establish your hunting grounds.} Look at the map and pick a district in which to place your hunting grounds. Decide how to deal with the faction that claims that area.
		\begin{itemize}
		\item Pay them 1 \gameterm{coin} .
		\item Pay them 2 \gameterm{coin} . Get \textbf{+1 status}.
		\item Pay nothing. Get \textbf{-1 status}.
		\end{itemize}
	\item \textbf{Choose a special ability.} They’re in the gray column in the middle of the crew sheet. If you can’t decide, choose the first ability on the list. It’s placed there as a good first option.
	\item \textbf{Assign crew upgrades.} Your crew has two upgrades pre-selected. Choose two more. If your crew has a cohort, follow the procedure to create it. Record the faction status changes due to your upgrades:
		\begin{itemize}
		\item One faction helped you get an upgrade. Take \textbf{+1 status} with them. Or spend 1 \gameterm{coin}  for \textbf{+2 status} instead.
		\item One faction was harmed when you got an upgrade. Take \textbf{-2 status} with them. Or spend 1 \gameterm{coin}  for \textbf{-1 status} instead.
		\end{itemize}
	\item \textbf{Choose a favorite contact.} Mark the one who is a close friend, long-time ally, or partner in crime. Record the faction status changes related to your contact:
		\begin{itemize}
		\item One faction is friendly with your contact. Take \textbf{+1 status} with them.
		\item One faction is unfriendly with your contact. Take \textbf{-1 status} with them.
		\end{itemize}
		At your option, increase the intensity of the factions’ relationship with your contact and take \textbf{+2} and \textbf{-2 status}, instead.
\end{enumerate}

\chapter{Crew playbook}

\textbf{Short description of Crew}

You’re professional murderers---death is your business.

\textbf{Add xp trigger. When you play this crew, you earn xp when you execute a successful operation of a few specific types.} Some examples include: murder, ransom, battle extortion, sabotage, product supply or sale, espionage, theft, acquiring new contraband or sources.

\emph{Add questions to personalize the crew. How is the crew distinguished from other similar crews?}

\section{Starting upgrades}
\begin{itemize}
	\item First upgrade
	\item Second upgrade
\end{itemize}

\section{Favored Operations}
A list of operation types that the crew may perform. Some possibilities include: murder, ranson, extortion, sabotage, aquisition, consecration, sacrifice, sale, supply, show of force, sociailize, burglary, espionage, robbery, smuggling arms, smuggling contraband, or smuggling passengers.

\section{Contacts}

A list of five possible contacts for thew crew, along with descriptions of each. Some possibilities include: a gang boss, a deal broker, a noble, a bounty hunter, a merchant, a blacksmith, a physicker, a ward boss, a taver owner, an academic, an occultist, a magistrate, a dillettante, an explorer, a collector, an arms dealer, a drug dealer, an anarchist, or a dock worker.

Questions can inlclue how you know the friend, what they’ve done for you,  what you do for them, and what kind of relationship you have.

\section{Crew upgrades}

A list of five upgrades specific to the crew. Some examples:

\begin{itemize}
	\item \textbf{Rigging:} You get 2 free load in two of the equipment categories (weapons, implements, supplies, gear, documents, and tools).
	\item \textbf{Prison Contacts:} Your Tier is effectively +1 higher in prison. \emph{This counts for any Tier-related element in prison, including the incarceration roll.}
	\item \textbf{Elite Cohorts (specify type):} All of your cohorts with the specified type get \textbf{+1d} to quality rolls for related actions.
	\item \textbf{Hardened:} Each PC gets \textbf{+1 }\gameterm{trauma} \textbf{ box}. This costs three upgrades to unlock, not just one. \emph{This may bring a PC with 4 }\gameterm{trauma} \emph{ back into play if you wish.}
	\item \textbf{Composed:} Each PC gets \textbf{+1 stress box}. This costs three upgrades to unlock, not just one.
	\item \textbf{Underground Maps and Passkeys:} You have easy passage through the underground canals, tunnels, and basements of the city.
	\item \textbf{Camouflage:} Your vehicles are perfectly concealed when at rest. They blend in as part of the environment, or as an uninteresting civilian vehicle (your choice).
\end{itemize}

\subsection{Make a Claim Map for the Crew. Sample claims include:}

\gameterm{Turf: } You require one less Rep to advance in tier (max 6).

\gameterm{City Records:}  You get \textbf{+1d} to the \textbf{engagement roll} for \textbf{stealth} plans. \emph{You can use blueprints and other documents to determine a good approach for infiltrations.}

\gameterm{Cover Identities: }  You get \textbf{+1d} to the \textbf{engagement roll} for \textbf{deception} and **social plans. \emph{False identities help confuse the opposition.}

\gameterm{Cover Operation: }  You get \textbf{-2} \gameterm{heat}  per score. \emph{The cover of a legitimate operation helps deflect some of the heat from law enforcement.}

\gameterm{Envoy:}  You get \textbf{+2} \gameterm{coin}  in \textbf{payoff} for scores that involve high-class clients. \emph{This well-connected liaison will help arrange for a better payoff from rich clients.}

\gameterm{Fixer:}  You get \textbf{+2} \gameterm{coin}  in \textbf{payoff} for scores that involve lower-class clients. \emph{This well-respected agent will help arrange for a better payoff from poorer clients.}

\gameterm{Infirmary:}  You get \textbf{+1d} to healing treatment rolls. \emph{The infirmary also has beds for long-term convalescence.}

\gameterm{Informants:}  You get \textbf{+1d} to gather information for a score. \emph{Your eyes and ears on the streets are always on the lookout for new targets.}

\gameterm{Protection Racket:}  Any time during downtime, roll dice equal to your Tier. You earn \gameterm{coin}  equal to the highest result, minus your \gameterm{heat} . \emph{Some of the locals are terrified of you and will gladly pay for “protection.”}

\gameterm{Training Rooms:}  Your cohorts of a specific type get \textbf{+1 scale}. \emph{Extra training enables them to fight like a larger gang.}

\gameterm{Vice Den:}  Any time during downtime, roll dice equal to your Tier. You earn \gameterm{coin}  equal to the highest result, minus your \gameterm{heat} .

\gameterm{Victim Trophies:}  You get \textbf{+1 rep} per score. \emph{Word of your grisly “collection” gets around, and your boldness boosts your rep in the underworld.}

\gameterm{Fighting Pits }  During downtime, roll dice equal to your Tier. You earn \gameterm{coin}  equal to the highest result, minus your \gameterm{heat} . \emph{The locals love to gamble away their hard-won coin on the blood-sports you host.}

\gameterm{Street Fence:}  You get \textbf{+2} \gameterm{coin}  in your payoff for scores that involve lower-class targets. \emph{An expert can find the treasure amid the trash you loot from your poorer victims.}

\gameterm{Terrorized Citizens:}  You get \textbf{+2} \gameterm{coin}  in your payoff for scores that involve battle or extortion. \emph{The frightened locals offer you tribute whenever you lash out. They don’t want to be next.}

\gameterm{Warehouses:}  You get \textbf{+1d} to \textbf{acquire asset} rolls. \emph{You have space to hold all the various spoils you end up with after your battles. It can be useful on its own or for barter when you need it.}

\gameterm{Ancient Altar: }  You get \textbf{+1d} to the \textbf{engagement roll} for \textbf{occult} plans. \emph{Its blessing is with you.}

\gameterm{Ancient Tower:}  You get \textbf{+1d} to \gameterm{Consort}  with arcane entities on-site.

\gameterm{Offertory:}  You get \textbf{+2} \gameterm{coin}  in your payoff for scores that involve occult operations. \emph{The frightened locals offer you tribute when you perform your dark practices. They don’t want to be next.}

\gameterm{Local Graft:}  You get \textbf{+2} \gameterm{coin}  in \textbf{payoff} for scores that involve a show of force or socializing. \emph{A few city officials share bribe money with those who show that they’re players on the scene.}

\gameterm{Lookouts: }  You get \textbf{+1d} to \gameterm{Hunt}  or \gameterm{Survey}  on your turf.

\gameterm{Luxury Venue: }  \textbf{+1d} to \gameterm{Consort}  and \gameterm{Sway}  rolls on-site. \emph{Silks, paintings, and crystal impress the clientele.}

\gameterm{Surplus Cache:}  You get \textbf{+2} \gameterm{coin}  in \textbf{payoff} for scores that involve product sale or supply. \emph{You have an abundance of product, which pads your pockets every now and then.}

\gameterm{Covert Drop:}  You get \textbf{+2} \gameterm{coin}  in \textbf{payoff} for scores that involve espionage or sabotage. \emph{The perfect hidden exchange point is worth the extra coin to discerning clientele.}

\gameterm{Interrogation Chamber:}  You get \textbf{+1d} to \gameterm{Command}  and \gameterm{Sway}  on-site. \emph{Grisly business, but effective.}

\gameterm{Loyal Fence:}  You get \textbf{+2} \gameterm{coin}  in \textbf{payoff} for scores that involve burglary or robbery. \emph{It requires a skilled eye and good contacts to move stolen goods.}

\gameterm{Secret Pathways: }  You get \textbf{+1d} to the \textbf{engagement roll} for \textbf{stealth} plans. \emph{You might have access to long-forgotten underground canals, rooftop walkways, or some other route of your choosing.}

\gameterm{Tavern:}  You get \textbf{+1d} to \gameterm{Consort}  and \gameterm{Sway}  rolls on-site. \emph{Some booze and friendly conversation can go a long way.}

\gameterm{Side Business:}  Any time during downtime, roll dice equal to your Tier. You earn \gameterm{coin}  equal to the highest result, minus your \gameterm{heat} . \emph{What kind of legitimate business is this? How do you get paid in secret?}

\section{Crew special abilities}

Create seven special abilities for your crew. Here are several examples.

\subsection{Deadly}

Each PC may add +1 action rating to \gameterm{Hunt} , \gameterm{Prowl} , or \gameterm{Skirmish}  (up to a max rating of 3).

\begin{quote}
	Each player may choose the action they prefer (you don’t all have to choose the same one). If you take this ability during initial character and crew creation, it supersedes the normal starting limit for action ratings.
\end{quote} 

\subsection{Death veil}

Due to hard-won experience or occult ritual, you don’t take extra \gameterm{heat}  when killing is involved on a score.

\subsection{No traces}

When you keep an operation quiet or make it look like an accident, you get half the \gameterm{rep}  value of the target (round up) instead of zero. When you end downtime with zero \gameterm{heat} , take \textbf{+1} \gameterm{rep} .

\begin{quote}
	There are many clients who value quiet operations. This ability rewards you for keeping a low-profile.
\end{quote} 

\subsection{Patron}

When you advance your \textbf{Tier}, it costs half the \gameterm{coin}  it normally would.

\begin{quote}
	Who is your patron? Why do they help you?
\end{quote} 

\subsection{Predators}

When you use a stealth or deception plan to commit murder, take \textbf{+1d} to the \textbf{engagement roll}.

\begin{quote}
	This ability applies when the goal is murder. It doesn’t apply to other stealth or deception operations you attempt that happen to involve killing.
\end{quote} 

\subsection{Fiends}

Fear is as good as respect. You may count each \gameterm{wanted level}  as if it were \textbf{turf}.

\begin{quote}
	The maximum \gameterm{wanted level}  is 4. Regardless of how much turf you hold (from this ability or otherwise) the minimum \gameterm{rep}  cost to advance your Tier is always 6.
\end{quote} 

\subsection{Forged in the fire}

Each PC has been toughened by cruel experience. You get \textbf{+1d} to \textbf{resistance} rolls.

\begin{quote}
	This ability applies to PCs in the crew. It doesn’t confer any special toughness to your cohorts.
\end{quote} 

\subsection{War dogs}

When you’re at war (-3 faction status), your crew does not suffer -1 hold and PCs still get two downtime activities, instead of just one.

\subsection{Anointed}

You gain \textbf{+1d} to \textbf{resistance} rolls against supernatural threats. You get \textbf{+1d} to healing rolls when you have supernatural harm.

\subsection{Accord}

Sometimes friends are as good as territory. You may treat up to three \textbf{+3 faction} statuses\textbf{ you hold as if they are }turf**.

\begin{quote}
	If your status changes, you lose the turf until it becomes +3 again. Regardless of how much turf you hold (from this ability or otherwise) the minimum \gameterm{rep}  cost to advance your Tier is always 6.
\end{quote} 

\subsection{High society}

It’s all about who you know. Take \textbf{-1 }\gameterm{heat} \textbf{ }during downtime and \textbf{+1d} to \textbf{gather information} about the city’s elite.

\subsection{Pack rats}

Your lair is a jumble of stolen items. When you roll to \textbf{acquire an asset}, take \textbf{+1d}.

\begin{quote}
	This ability might mean that you actually have the item you need in your pile of stuff, or it could mean you have extra odds and ends to barter with.
\end{quote} 

\subsection{Second story}

When you execute a clandestine infiltration, you get \textbf{+1d} to the \textbf{engagement roll}.

\subsection{Synchronized}

When you perform a \textbf{group action}, you may count multiple 6s from different rolls as a \gameterm{critical}  success.

\begin{quote}
	For example, Lyric leads a group action to \gameterm{Attune}  to the ghost field to overcome a magical ward on the Dimmer Sisters’ door. Emily, Lyric’s player, rolls and gets a \gameterm{6} , and so does Matt! Because the crew has Synchronized, their two separate \gameterm{6} s count as a \gameterm{critical}  success on the roll.
\end{quote} 

\subsection{All hands}

During \textbf{downtime}, one of your \textbf{cohorts} may perform a downtime activity for the crew to \textbf{acquire an asset}, \textbf{reduce} \gameterm{heat} , or work on a \textbf{long-term project}.

\subsection{Just passing through}

During \textbf{downtime}, take \textbf{-1 }\gameterm{heat} . When your \gameterm{heat}  is 4 or less, you get \textbf{+1d} to deceive people when you pass yourselves off as ordinary citizens.

\part{The Score}

\emph{Murder for hire, brutal extortion, dark rituals, illicit deals, smuggling runs, thievery in the shadows---the only chances left for those pushed to the margins and denied the privileges of the corrupt and predatory elite.}

In Blades in the Dark, we play to find out if a fledgling crew of characters can prosper in the underworld---and that prosperity depends upon their criminal endeavors, which we call \textbf{scores}.

A score is a single operation with a particular goal: \emph{burgle a Lord’s manor, assassinate the diplomat, smuggle a strange artifact into the city}, etc. Usually, a score will fall into one of three categories:

\begin{itemize}
	\item A \textbf{criminal activity}, determined by your crew type. An assassination, burglary, illicit vice deal, etc.
	\item Seizing a \textbf{claim} that you choose from your crew’s \textbf{claim map}. Claims help your crew grow and develop.
	\item A \textbf{special mission} or goal determined by the players (like getting a rare artifact to empower one of the Whisper’s rituals).
\end{itemize}

A score can be long and involved or short and sweet. There might be lots of rolls and trouble, or just a few actions to resolve it. \emph{Play to find out what happens!} A score doesn’t need to fill one session of play every time. Let it be however long it is.

The PCs can set up a new score by choosing a target (from their claims or the faction list, for example), by approaching a potential client and asking for work, or by being contacted by an NPC who needs to hire a crew for a job.

A score consists of a few key elements, detailed in this chapter: \textbf{planning \& engagement}, \textbf{flashbacks}, and \textbf{teamwork}.

\chapter{Planning \& engagement}

Your crew spends time planning each score. They huddle around a flickering lantern in their lair, looking at scrawled maps, whispering plots and schemes, bickering about the best approach, lamenting the dangers ahead, and lusting after stacks of coin.

But you, the players, don’t have to do the nitty-gritty planning. The characters take care of that, off-screen. All you have to do is choose what \textbf{type of plan} the characters \emph{have already made}. There’s no need to sweat all the little details and try to cover every eventuality ahead of time, because the \textbf{engagement roll} (detailed on the next page) ultimately determines how much trouble you’re in when the plan is put in motion. \emph{No plan is ever perfect.} You can’t account for everything. This system assumes that there’s always some unknown factors and trouble---major or minor---in every operation; you just have to make the best of it.

There are six different plans, each with a missing \textbf{detail} you need to provide (see the list below). To “plan an operation,” simply choose the plan and supply the detail. Then the GM will \textbf{cut to the action} as the first moments of the operation unfold.

@TODO Add table p79

\section{The Detail}

When you choose a plan, you provide a missing \textbf{detail}, like the point of attack, social connection, etc. If you don’t know the detail, you can \textbf{gather information} in some way to discover it.

\section{Item Loadouts}

After the plan and detail are in place, each player chooses their character’s \textbf{load}. This indicates how much stuff they’re carrying on the operation. They don’t have to select individual items---just the maximum amount they’ll have access to during the action.

\section{Engagement Roll}

Once the players choose a plan and provide its detail, the GM cuts to the action---describing the scene as the crew starts the operation and encounters their first obstacle. But how is this established? The way the GM describes the starting situation can have a huge impact on how simple or troublesome the operation turns out to be. Rather than expecting the GM to simply “get it right” each time, we use a dice roll instead. This is the \textbf{engagement roll}.

The engagement roll is a \textbf{fortune roll}, starting with \textbf{1d for sheer luck}. Modify the dice pool for any major advantages or disadvantages that apply.

\subsection{Major Advantages / Disadvantages}

\begin{itemize}
	\item Is this operation particularly bold or daring? Take \textbf{+1d}. Is this operation overly complex or contingent on many factors? Take \textbf{-1d}.
	\item Does the \textbf{plan’s detail} expose a vulnerability of the target or hit them where they’re weakest? Take \textbf{+1d}. Is the target strongest against this approach, or do they have particular defenses or special preparations? Take \textbf{-1d}.
	\item Can any of your \textbf{friends or contacts }provide aid or insight for this operation? Take \textbf{+1d}. Are any \textbf{enemies or rivals} interfering in the operation? Take \textbf{-1d}.
	\item Are there any \textbf{other elements} that you want to consider? Maybe a lower-Tier target will give you +1d.  Maybe a higher-Tier target will give you -1d. Maybe there’s a situation in the district that makes the operation more or less tricky.
\end{itemize}

The engagement roll assumes that the PCs are approaching the target as intelligently as they can, given the plan and detail they provided, so we don’t need to play out tentative probing maneuvers, special precautions, or other ponderous non-action. The engagement roll covers all of that. The PCs are already in action, facing the first obstacle---up on the rooftop, picking the lock on the window; kicking down the door of the rival gang’s lair; maneuvering to speak with a Lord at the masquerade party; etc.

Don’t make the engagement roll and then describe the PCs \emph{approaching} the target. It’s the approach that the engagement roll resolves. Cut to the action that results \emph{because of} that initial approach---to the first serious obstacle in their path.

@TODO add table p81

\begin{quote}
	The first obstacles at the witches’ house are their cunning locks and magical traps. The engagement roll puts us on the roof outside a window, as the PCs attempt to silently and carefully break into the attic.
\end{quote} 

\begin{quote}
	The PCs have kicked down the door and swarmed into the front room of the gang’s lair, weapons flashing, into the swirl of the melee with the first guards.
\end{quote} 

\begin{quote}
	The PCs have socialized politely at the party, maneuvering into position to have a private word with a powerful Lord. As a group of young nobles leave his side, the PCs step up and engage him in conversation.
\end{quote} 

If the players want to include a special preparation or clever setup, they can do so with \textbf{flashbacks} during the score. This takes some getting used to. Players may balk at first, worried that you’re skipping over important things that they want to do. But jumping straight into the action of the score is much more effective once you get used to it. When they see the situation they’re in, their “planning” in flashbacks will be focused and useful, rather than merely speculations on circumstances and events that might not even happen.

\subsection{Outcomes}

The outcome of the engagement roll determines the \textbf{position} for the PCs’ initial actions when we cut to the score in progress. A \gameterm{1-3}  means a desperate position. A \gameterm{4/5}  is a risky position. A \gameterm{6}  yields a controlled position. And a \gameterm{critical}  carries the action beyond the initial obstacle, deeper into the action of the score.

No matter how low-Tier or outmatched you are, a desperate position is the worst thing that can result from the plan + detail + engagement process. It’s designed this way so the planning process matters, but it doesn’t call for lots of optimization or nitpicking. Even if you’re reckless and just dive in and take your chances, you can’t get too badly burned. Plus, you might even want those desperate rolls to generate more xp for the PCs, which helps to bootstrap starting characters into advancement.

When you describe the situation after the roll, use the details of the target to paint a picture of the PCs’ position. How might the strange, occult gang present a desperate position for burglars? How might the violent and ruthless butchers present a risky threat to assaulting thugs? How might the vain and pompous Lord present a controlled opportunity for a manipulative scoundrel? Use this opportunity to show how the PCs’ enemies are dangerous and capable---don’t characterize a bad engagement roll as a failure by the PCs, or they won’t trust the technique in the future. Sure, things are starting out desperate here against the creepy occultists, but you’re just the type of characters who are daring enough to take them on. Let’s get to it.

\subsection{How long does it last?}

The engagement roll determines the starting position for the PCs’ actions. How long does that hold? Does the situation stay desperate? No. Once the initial actions have been resolved, you follow the normal process for establishing position for the rest of the rolls during the score. The engagement roll is a quick short-hand to kick things off and get the action started---it doesn’t have any impact after that.

\section{Linked Plans}

Sometimes an operation seems to call for a couple of plans linked together. A common scenario is a team that wants a two-pronged approach. “You create a diversion at the tavern, and when they send thugs over there, we’ll break into their lair.” There are two ways to handle this.

\begin{enumerate}
	\item The diversion is a \textbf{setup maneuver} that a team member performs as part of the plan. A successful setup maneuver can improve \textbf{position} for teammates (possibly offsetting a bad engagement roll) or give increased effect.  An unsuccessful setup maneuver might cause trouble for the second part of the plan---an easy consequence is to give the engagement roll -1d. \emph{If it makes sense, the team member who performed the setup can drift back into the main operation and join the team later so they don’t have to sit out and wait.}
	\item The diversion is its own plan, engagement, and operation, whose outcome creates the opportunity for a future plan. Use this option when the first part of the plan is required for the next part to happen at all. For example, you might execute a stealth plan to steal an artifact from the Museum of the Ancients, then later use that artifact in an occult plan to consecrate a temple for your forgotten god. In this case, you go into downtime (and payoff, \gameterm{heat} , etc.) after the first part of the plan, as normal.
\end{enumerate}

Either approach is fine. It’s usually a question of interest. Is the linked plan idea interesting enough on its own to play out moment by moment? Is it required for the second plan to make sense? If so, make it a separate operation. If not, just use a setup maneuver.

\section{Flashbacks}

The rules don’t distinguish between actions performed in the present moment and those performed in the past. When an operation is underway, you can invoke a \textbf{flashback} to roll for an action in the past that impacts your current situation. Maybe you convinced the district Watch sergeant to cancel the patrol tonight, so you make a \gameterm{Sway}  roll to see how that went.

The GM sets a \textbf{stress cost} when you activate a flashback action.

\begin{itemize}
	\item \gameterm{0 Stress:}  An ordinary action for which you had easy opportunity. \emph{The Cutter }\gameterm{Consorted} \emph{ with her friend to agree to arrive at the dice game ahead of time, to suddenly spring out as a surprise ally.}
	\item \gameterm{1 Stress:}  A complex action or unlikely opportunity. \emph{The Hound }\gameterm{Finessed} \emph{ his pistols into a hiding spot near the card table so he could retrieve them after the pat-down at the front door.}
	\item \gameterm{2 (or more) Stress:}  An elaborate action that involved special opportunities or contingencies. \emph{The Whisper has already \gameterm{Studied}  the history of the property and learned of a ghost that is known to haunt its ancient canal dock---a ghost that can be compelled to reveal the location of the hidden vault.}
\end{itemize}

After the stress cost is paid, a flashback action is handled just like any other action. Sometimes it will entail an action roll, because there’s some danger or trouble involved. Sometimes a flashback will entail a fortune roll, because we just need to find out how well (or how much, or how long, etc.). Sometimes a flashback won’t call for a roll at all because you can just pay the stress and it’s accomplished.

If a flashback involves a \textbf{downtime} activity, pay 1 \gameterm{coin}  or 1 \gameterm{rep}  for it, instead of stress.

One of the best uses for a flashback is when the \textbf{engagement roll} goes badly. After the GM describes the trouble you’re in, you can call for a flashback to a special preparation you made, “just in case” something like this happened. This way, your “flashback planning” will be focused on the problems that \emph{do }happen, not the problems that \emph{might} happen.

\subsection{Limits of flashbacks}

A flashback isn’t time travel. It can’t “undo” something that just occurred in the present moment. For instance, if an Inspector confronts you about recent thefts of occult artifacts when you’re at the Lady’s party, you can’t call for a flashback to assassinate the Inspector the night before. She’s here now, questioning you---that’s established in the fiction. You \emph{can} call for a flashback to show that you intentionally tipped off the inspector so she would confront you at the party---so you could use that opportunity to impress the Lady with your aplomb and daring.

\subsection{flashback examples}

\begin{quote}
	“I want to have a flashback to earlier that night, where I sneak into the stables and feed fireweed to all their goats so they’ll go berserk and create a distraction for our infiltration.”
\end{quote} 

\begin{quote}
	“Ha! Nice. Okay, that’s seems a bit tricky, dealing with ornery goats and all... 1 stress.”
\end{quote} 

\begin{quote}
	“Should I roll Prowl to sneak in and plant it?”
\end{quote} 

\begin{quote}
	“Nah. Their goat stable security amounts to a stable boy who is usually asleep anyway. You can easily avoid their notice.”
\end{quote} 

\begin{quote}
	“So it just works?”
\end{quote} 

\begin{quote}
	“Eh... not so fast. When you want the distraction to hit, let’s make a fortune roll to see how crazy the Fireweed Goat Maneuver gets. Three dice.”
\end{quote} 

---

\begin{quote}
	“The engagement roll is... a \gameterm{2} . Looks like a desperate situation for you! Hmmm. Okay, so you’re inside the gang’s compound at the docks, slipping up through the shadows next to some huge metal storage tanks. But then all the electric lights come on. The big metal warehouse door rolls open, and you hear a heavy wagon coming in through the gate. Looks like they’re getting a delivery right now, and a bunch of gang members are out to receive it. They’re about to be on top of you. What do you do?”
\end{quote} 

\begin{quote}
	“Hang on, I want to have a flashback.”
\end{quote} 

\begin{quote}
	“Okay, for what?”
\end{quote} 

\begin{quote}
	“Uh. Something... helpful? Damn, I don’t know what that would be. Anyone have ideas?”
\end{quote} 

\begin{quote}
	“Oh, what if you Consorted with your docker friends yesterday and they blabbed about this delivery, so we rigged it with a bomb.”
\end{quote} 

\begin{quote}
	“Oh man, that’s hilarious. But kind of nuts. I guess 2 stress for that?”
\end{quote} 

\begin{quote}
	“Sounds good. But let’s make that Consort roll and see if your docker friends made any demands or complicated anything for you. Then we need to find out how well this bomb works. Who was in charge of that?”
\end{quote} 

\begin{quote}
	“I did it. I’ll roll Tinker to set the fuse just right. Hopefully.”
\end{quote} 

\section{Giving up on a score}

When you give up on a score, you go into \textbf{downtime}. Follow the phases for downtime presented in the next chapter. You’ll usually have zero \textbf{payoff}, since you didn’t accomplish anything. You’ll still face \gameterm{heat}  and \textbf{entanglements} as usual.

\chapter{Teamwork}

\section{Teamwork}

When the team of PCs works together, the characters have access to four special \textbf{teamwork maneuvers}. They’re listed at the bottom of the character playbook sheets to help remind the players of them. The four maneuvers are:

\begin{itemize}
	\item \textbf{Assist} another PC who’s rolling an action.
	\item Lead a \textbf{group action}.
	\item \textbf{Protect} a teammate.
	\item \textbf{Set up} a character who will follow through on your action.
\end{itemize}

\subsection{Assist}

When you assist another player who’s rolling, describe what your character does to help. Take 1 stress and give them \textbf{+1d} to their roll. You might also suffer any consequences that occur because of the roll, depending on the circumstances. Only one character may assist a given roll.\emph{ If you really want to help and someone else is already assisting, consider performing a \textbf{setup} maneuver instead.}

A character may assist a group action, but only if they aren’t taking part in it directly. You decide which character in the group action gets the bonus die.

\subsection{Lead a group action}

When you lead a group action, you coordinate multiple members of the team to tackle a problem together. Describe how your character leads the team in a coordinated effort. Do you bark orders, give subtle hand signals, or provide charismatic inspiration?

Each PC who’s involved makes an \textbf{action roll} (using the same action) and \textbf{the team counts the single best result} as the overall effort for everyone who rolled. However, the character leading the group action takes \textbf{1 stress} for each PC that rolled\gameterm{ 1-3}  as their best result.

\begin{quote}
	This is how you do the “we all sneak into the building” scene. Everyone who wants to sneak in rolls their Prowl action, and the best result counts for the whole team. The leader suffers stress for everyone who does poorly. It’s tough covering for the stragglers.
\end{quote} 

\textbf{The group action result covers everyone who rolled.} If you don’t roll, your character doesn’t get the effects of the action.

Your character doesn’t have to be especially skilled at the action at hand in order to lead a group action. This maneuver is about leadership, not necessarily about ability. You can also lead your crew’s \textbf{cohorts} with a group action. Roll \gameterm{Command}  if you direct their efforts, or roll the appropriate action rating if you participate alongside them. The cohort rolls its \textbf{quality} level.

\subsection{Protect}

You step in to face a consequence that one of your teammates would otherwise face. \textbf{You suffer it instead of them. You may roll to resist it as normal.} Describe how you intervene.

\begin{quote}
	This is how you do the “I’ll dive in front of the bullet” You cover for a teammate, suffering any harm or consequences that still linger after you’ve rolled to resist. It hurts, cost stress, and may leave you in a bad spot. But hey, you’re a hero.
\end{quote} 

\subsection{Set up}

When you perform a setup action, you have an indirect effect on an obstacle. If your action has its intended result, any member of the team who follows through on your maneuver gets \textbf{+1 effect level} or \textbf{improved position} for their roll. You choose the benefit, based on the nature of your setup action.

\begin{quote}
	This is how you do the “I’ll create a distraction” scene. You roll \gameterm{Sway}  to distract a constable with your charms, then any teammate who follows through with a \gameterm{Prowl}  action to sneak past him can get improved position. It’s less risky since you’re drawing the guard’s attention.
\end{quote} 

This is a good way to contribute to an operation when you don’t have a good rating in the action at hand. A clever setup action lets you help the team indirectly. Multiple follow-up actions may take advantage of your setup (including someone \textbf{leading} a group action) as long as it makes sense in the fiction.

Since a setup action can increase the effect of follow-up actions, it’s also useful when the team is facing tough opposition that has advantages in quality, scale, and/or potency. Even if the PCs are reduced to zero effect due to disadvantages in a situation, the setup action provides a bonus that allows for limited effect.

\begin{quote}
	The PCs are facing a heavily armored carriage that’s immune to their weapons. Aldo uses \gameterm{Wreck}  as a setup action to pry some of the armor loose with his crowbar, giving follow up actions +1 effect---going from zero to limited effect.
\end{quote} 

\section{Do We Have to Use Teamwork?}

Teamwork maneuvers are options, not requirements. Each character can still perform solo actions as normal during an operation. If your character can’t communicate or somehow coordinate with the rest of the team, you can’t use or benefit from any teamwork maneuvers.

\part{Downtime}

After the crew finishes a score (succeed or fail), they take time to recover, regroup, and prepare for the next operation. This phase of the game is called \textbf{downtime}.

Downtime fulfills two purposes in the game:

\begin{itemize}
	\item First, it’s a break for the players. During the action of the score, the PCs are always under threat, charging from obstacle to obstacle in a high-energy sequence. Downtime gives them a reprieve so they can catch their breath and relax a bit---focus on lower-energy, quieter elements of the game, as well as explore personal aspects of their characters.
	\item Second, the shift into a new phase of the game signals a shift in which mechanics are needed. There are special rules that are only used during the downtime phase, so they’re kept “out of the way” during the other parts of play. When we shift into downtime, we take out a different toolbox and resolve downtime on its own terms, then shift back into the more action-focused phases of the game afterwards.
\end{itemize}

Downtime is divided into four parts, which are resolved in order:
\begin{enumerate}
	\item \gameterm{Payoff. } The crew receives their rewards from a successfully completed score.
	\item \gameterm{Heat. } The crew accumulates suspicion and attention from the law and the powers-that-be in the city as a result of their last score.
	\item \gameterm{Entanglements.}  The crew faces trouble from the rival factions, the law, and the haunted city itself.
	\item \gameterm{Downtime Activities.}  The PCs indulge their vices to remove stress, work on long-term projects, recover from injuries, etc.
\end{enumerate}

After the downtime activities are resolved, the game returns to free play, and the group can move toward their next score.

\chapter{Payoff}

After a score, the PCs take stock of their income from the operation. A successful score generates both \gameterm{rep}  and \gameterm{coin} .

The crew earns 2 \gameterm{rep}  per score by default. If the target of the score is higher Tier than you, take \textbf{+1 }\gameterm{rep} \textbf{ per Tier higher}. If the target of the score is lower Tier, you get \textbf{-1 }\gameterm{rep} \textbf{ per Tier lower} (minimum zero).

\begin{quote}
	If your crew is Tier I and you pull off a successful score against a Tier III target, you earn 4 \gameterm{rep}  (2 \gameterm{rep} , +2 \gameterm{rep}  for a target two tiers above you). If your crew is Tier III and you complete a score against a Tier I target, you earn 0 \gameterm{rep}  (2 rep, -2 rep for the lower Tier target).
\end{quote} 

If you keep the operation completely quiet so no one knows about it, you earn zero \gameterm{rep} . Mark the rep on the \gameterm{rep}  tracker on the crew sheet.

The crew earns \gameterm{coin}  based on the nature of the operation and/or any loot they seized:

\begin{itemize}
	\item \gameterm{2 coin:}  A minor job; several full purses.
	\item \gameterm{4 coin:}  A small job; a strongbox.
	\item \gameterm{6 coin:}  A standard score; decent loot.
	\item \gameterm{8 coin:}  A big score; serious loot.
	\item \gameterm{10+ coin:}  A major score; impressive loot.
\end{itemize}

Record the \gameterm{coin}  on the crew sheet, or divvy it up among the crew members as you see fit.

Most districts have crime bosses that expect smaller crews to pay a tithe from their scores. Ask the GM if there’s a boss that you should be paying. \textbf{Subtract \gameterm{coin}  equal to your crew Tier +1} when you pay a tithe to a boss or larger organization. \emph{If you’re supposed to be paying off a boss, but you don’t, start a clock for that boss’s patience running out. Tick it whenever you don’t pay. Every time it fills up, lose 1 faction status with them.}

You can set the scene and play out a meeting with a client or patron who’s paying the crew if there’s something interesting to explore there. If not, just gloss over it and move on to the next part of downtime.

GM, definitely don’t screw around with the players when it comes to the payoff. Don’t say that the client lied and there’s no reward. Or that the meeting for the payment is actually a trap, or whatever. These types of things are staples of crime fiction, but in \emph{Blades}, the PCs have enough problems coming at them from every direction already. When it comes to getting paid, just give them what they earned.

\chapter{Heat}

The city is full of prying eyes and informants. Anything you do might be witnessed, and there’s always evidence left behind. To reflect this, your crew acquires \gameterm{heat}  as they commit crimes. After a score or conflict with an opponent, your crew takes \gameterm{heat}  according to the nature of the operation:

\begin{itemize}
	\item \gameterm{0 heat:}  Smooth \& quiet; low exposure.
	\item \gameterm{2 heat:}  Contained; standard exposure.
	\item \gameterm{4 heat: } Loud \& chaotic; high exposure.
	\item \gameterm{6 heat:}  Wild; devastating exposure.
\end{itemize}

Add +1 \gameterm{heat}  for a high-profile or well-connected target. Add +1 \gameterm{heat}  if the situation happened on hostile turf. Add +1 \gameterm{heat}  if you’re at war with another faction. Add +2 \gameterm{heat}  if killing was involved (whether the crew did the killing or not---bodies draw attention).

You mark \gameterm{heat}  levels on the \gameterm{heat}  tracker on the crew sheet.

@TODO add table p91

When your \gameterm{heat}  level reaches 9, you gain a \gameterm{wanted level}  and clear your \gameterm{heat}  (any excess \gameterm{heat}  “rolls over,” so if your \gameterm{heat}  was 7 and you took 4 \gameterm{heat} , you’d reset with 2 \gameterm{heat}  marked).

The higher your \gameterm{wanted level} , the more serious the response when law enforcement takes action against you (they’ll send a force of higher \textbf{quality} and \textbf{scale}).

Also, your \gameterm{wanted level}  contributes to the severity of the \textbf{entanglements} that your crew faces after a score.

\section{Incarceration}

The only way to reduce your crew’s \gameterm{wanted level}  is through incarceration. When one of your crew members, friends, contacts---or a framed enemy---is convicted and incarcerated for crimes associated with your crew, your \gameterm{wanted level}  is reduced by 1 and you clear your \gameterm{heat} .

Incarceration may result from investigation and arrest by the officers, or because someone turns themselves in and takes the fall for the crew’s crimes.

The severity of the prison sentence depends on your \gameterm{wanted level} :

\begin{itemize}
	\item \gameterm{Wanted Level 4: } Life or execution.
	\item \gameterm{Wanted Level 3:}  A year or two.
	\item \gameterm{Wanted Level 2:}  Several months.
	\item \gameterm{Wanted Level 1:}  A month or two.
	\item \gameterm{Wanted Level 0:}  A few weeks. Or, the constables give you a beating to teach you a lesson (suffer level 3 harm, no resistance roll allowed---they keep going until you’re injured).
\end{itemize}

Incarceration is dehumanizing and brutal. The renown of your crew is your only real defense inside. When you serve your time, make an \textbf{incarceration roll} using your crew’s Tier as the dice pool.

@TODO add table p92

\section{Prison claims}

@TODO add table p93

\subsection{Allied claim}

One of your allies on the inside arranges for their faction to grant you a boon. Take a claim for your crew from a different crew type. You can’t take turf with this claim.

\subsection{Cell block control}

Your crew has a cell block under their total control---guards and all. You never take \gameterm{trauma}  from incarceration.

\subsection{Guard payoff}

You claim several prison guards on your payroll. Take +1d to your Tier roll when a member of your crew is incarcerated.

\subsection{Hardcase}

Your reputation as a tough inmate bolsters your crew’s image. When your crew advances Tier, it costs 2 fewer \gameterm{coins}  than it normally would.

\subsection{Parole influence}

Political pressures of various sorts can be applied to the magistrates and warden who oversee sentences for crimes. With this claim, you’re always able to arrange for a shorter prison stay---as if your \gameterm{wanted level}  was 1 lower. So, if your \gameterm{wanted level}  was 3 when you went in, you’d spend only several months behind bars (equivalent to level 2) instead of a full year.

\subsection{Smuggling}

You arrange smuggling channels inside. You have \textbf{+}2 load** while incarcerated, (starting from zero as a prisoner). If you take this claim twice, you’ll have 4 load while you’re serving time in Ironhook. Also, you may choose to carry \gameterm{coin}  in place of load for purposes of bribes or acquiring assets while in prison. You may reset the items in your prison loadout whenever your crew has downtime.

\chapter{Entanglements}

Your crew didn’t just spring into existence tonight. You have a complex history of favors, commitments, debts, and promises that got you where you are today. To reflect this, after each score, you roll dice to find out which \textbf{entanglement} comes calling. An entanglement might be a rival crew looking to throw their weight around (and demand some \gameterm{coin} ), an Investigator of the City Watch making a case against your crew (but ready for a bribe), or even the attention of a vengeful ghost.

After payoff and \gameterm{heat}  are determined, the GM generates an entanglement for the crew using the lists below. Find the column that matches the crew’s current \gameterm{heat}  level. Then roll a number of dice equal to their \gameterm{wanted level} , and use the result of the roll to select which sort of entanglement manifests. \emph{If \gameterm{wanted level}  is zero, roll two dice and keep the lowest result.}

@TODO add table p94

Bring the entanglement into play immediately, or hold off until an appropriate moment. For example, if you get the \emph{Interrogation} entanglement, you might wait until a PC indulges their vice, then say the costables picked them up when they were distracted by its pleasures.

Entanglements manifest fully before the PCs have a chance to avoid them. When an entanglement comes into play, describe the situation after the entanglement has manifested. The PCs deal with it from that point---they can’t intercept it and defuse it before it happens. The purpose of the mechanic is to abstract a lot of the complex stuff happening in the backgrounds of the characters’ lives in order to generate trouble for them. Entanglements are the cost of doing business in the underworld---a good crew learns to roll with the punches and pick their battles.

The entanglements are detailed on the following pages. Each has a list of potential ways for the PCs to be rid of it. If you want the entanglements to be a momentary problem for the crew, stick to the suggested methods to resolve them, and move on to the next part of downtime. If you want to dive in and explore the entanglement in detail, set the scene and play out the event in full, following the actions and consequences where they lead.

\section{Arrest}

An Inspector presents a case file of evidence to a magistrate, to begin prosecution of your crew. They send a detail to arrest you (a gang at least equal in \textbf{scale} to your \gameterm{wanted level} ). Pay them off with \gameterm{coin}  equal to your \gameterm{wanted level}  +3, hand someone over for arrest (this clears your \gameterm{heat} ), or try to evade capture.

\begin{quote}
	A truncheon bangs on the shutters of the window. “Alright then! Come on out and let’s go quietly now!” It sounds like the bald Sergeant. When you peek out, you see a detail of about twenty constables, all geared up for a fight. The Sergeant mumbles under his breath, so only you inside can hear: “Or perhaps I have the wrong address?” He clears his throat and waits for some coin to appear.
\end{quote} 

\section{Cooperation}

A +3 status faction asks you for a favor. Agree to do it, or forfeit 1 \gameterm{rep}  per Tier of the friendly faction, or lose \textbf{1 status} with them. If you don’t have a +3 faction status, you avoid entanglements right now.

\section{Demonic notice}

A demon approaches the crew with a dark offer. Accept their bargain, hide until it loses interest (forfeit3 \gameterm{rep} ), or deal with it another way.

\section{Flipped}

One of the PCs’ rivals arranges for one of your contacts, patrons, clients, or a group of your customers to switch allegiances due to the \gameterm{heat}  on you. They’re loyal to another faction now.

\section{Gang trouble}

One of your gangs (or other cohorts) causes trouble due to their flaw(s). You can lose face (forfeit\gameterm{ rep}  equal to your Tier +1), make an example of one of the gang members, or face reprisals from the wronged party.

\section{Interrogation}

The officers round up one of the PCs to question them about the crew’s crimes. \emph{How did they manage to capture you?} Either pay them off with 3 \gameterm{coin} , or they beat you up (\textbf{level 2 harm}) and you tell them what they want to know (\textbf{+3} \gameterm{heat} ). You can \textbf{resist} each of those consequences separately.

\begin{quote}
	Some players really hate it when their character gets captured! Just tell them that this is completely normal for a scoundrel of the underworld. You spend time in and out of jail, getting questioned and harassed by the law. It’s not the end of the world. But now that you’re here in the interrogation room, what kind of person are you? Do you talk? Do you stand up to them? Do you make a deal?
\end{quote} 

\section{Questioning}

The cops grab an NPC member of your crew or one of the crew’s contacts, to question them about your crimes. \emph{Who do they think is most vulnerable?} Make a \textbf{fortune roll} to see how much they talk (\gameterm{1-3:}  \textbf{+2} \gameterm{heat} , \gameterm{4/5:}  \textbf{+1} \gameterm{heat} ), or pay the constables off with 2 \gameterm{coin} .

>Roll 2d for a normal person to see how well they keep quiet. If they’re an experienced underworld type or some kind of tough, give them 3d or 4d instead. If they’re soft or if they have some loyalty to the law, give them 1d or 0d.

\section{Reprisals}

An enemy faction makes a move against you (or a friend, contact, or vice purveyor). Pay them (1 \gameterm{rep}  and 1 \gameterm{coin} ) per Tier of the enemy as an apology, allow them to mess with you or yours, or fight back and show them who’s boss.

\section{Rivals}

A neutral faction throws their weight around. They threaten you, a friend, a contact, or one of your vice purveyors. Forfeit (1 \gameterm{rep}  or 1 \gameterm{coin} ) per Tier of the rival, or stand up to them and lose\textbf{ 1 status}with them.

\section{Show of force}

A faction with whom you have a negative status makes a play against your holdings. Give them 1 \textbf{claim} or go to war (drop to -3 status). If you have no claims, \textbf{lose 1 hold} instead.

\section{Unquiet dead}

A rogue spirit is drawn to you---perhaps it’s a past victim? Acquire the services of an occultist to attempt to destroy or banish it, or deal with it yourself.

\begin{quote}
	They can hire an NPC by using the \gameterm{acquire asset} downtime activity. Roll the NPC’s quality level as a fortune roll to see how well they deal with the spirit.
\end{quote} 

\section{The usual suspects}

The cops grab someone in the periphery of your crew. One player volunteers a friend or vice purveyor as the person most likely to be taken. Make a \textbf{fortune roll} to find out if they resist questioning (\gameterm{1-3} : \textbf{+2 }\gameterm{heat} \textbf{, }4/5:\textbf{ level 2 harm}), or pay them off with 1 \gameterm{coin} .

\chapter{Downtime activities}

Between scores, your crew spends time at their liberty, attending to personal needs and side projects. These are called \textbf{downtime activities} (see the list below). During a downtime phase, each PC has time for \textbf{two downtime activities}. \emph{When you’re at war, each PC has time for only one.}

\begin{itemize}
	\item Acquire Asset
	\item Long-Term Project
	\item Recover
	\item Reduce Heat
	\item Train
	\item Indulge Vice
\end{itemize}

You may choose the same activity more than once. You can only attempt actions that you’re in a position to accomplish. If an activity is contingent on another action, resolve that action first.

A PC can make time for more than two activities, at a cost. \textbf{Each additional activity from the list costs} \textbf{1} \gameterm{coin}  or \textbf{1} \gameterm{rep} . This reflects the time and resulting resource drain while you’re “off the clock” and not earning from a score. When you complete a new score, you reset and get two “free” activities again.

Activities on the downtime list are limited; normal actions are not. During downtime, you can still go places, do things, make action rolls, gather information, talk with other characters, etc. In other words, only activities that are \emph{on the list} are limited.

For any downtime activity, take \textbf{+1d} to the roll if a \textbf{friend} or \textbf{contact} helps you. After the roll, you may spend \gameterm{coin}  after the roll to improve the result level. \textbf{Increase the result level by one for each }\gameterm{coin} \textbf{ spent.} So, a \gameterm{1-3}  result becomes a \gameterm{4}  or a \gameterm{5} , a \gameterm{4/5}  result becomes a \gameterm{6} , and a \gameterm{6}  becomes a \gameterm{critical} .

\textbf{GM:} If a player can’t decide which downtime activity to pick, offer them a long-term project idea. You know what the player is interested in and what they like. Suggest a project that will head in a fun direction for them.

\begin{quote}
	“Remember how you had that weird vision at the altar to the forgotten gods? Yeah, do you want to get to the bottom of that? Okay, start a long-term project---six segments---called... ‘Weird God Vibes.’ What do you do to work on that?”
\end{quote} 

\section{Acquire asset}

Gain temporary use of an \textbf{asset}:

\begin{itemize}
	\item One special \textbf{item} or set of common items (enough for a gang of your Tier scale).
	\item A \textbf{cohort} (an expert or gang).
	\item A \textbf{vehicle}.
	\item A \textbf{service}. Transport from a smuggler or driver, use of a warehouse for temporary storage, legal representation, etc.
\end{itemize}

“Temporary use” constitutes one significant period of usage that makes sense for the asset---typically the duration of one score. An asset may also be acquired for “standby” use in the future. You might hire a gang to guard your lair, for example, and they’ll stick around until after the first serious battle, or until a week goes by and they lose interest.

To acquire the asset, roll the crew’s Tier. The result indicates the \textbf{quality} of the asset you get, using the crew’s Tier as the base. \gameterm{1-3:}  Tier -1, \gameterm{4/5:}  Tier, \gameterm{6:}  Tier +1, \gameterm{critical:}  Tier +2. You can spend \gameterm{coin}  to raise the result of this roll beyond \gameterm{critical}  by spending \textbf{2} \gameterm{coin}  per additional Tier level added.

The GM may set a \textbf{minimum quality} level that must be achieved to acquire a particular asset. For example, if you want to get a set of Warden uniforms and masks, you’d need to acquire a Tier IV asset. A lower result won’t do.

If you acquire the same asset again, you get \textbf{+1d} to your roll. If you continue to re-acquire an asset every time it’s used, you can effectively rent it indefinitely.

Alchemicals, poisons, bombs, and dangerous gadgets are highly restricted. When you acquire one of these items (rather than crafting it yourself), you take \textbf{+2} \gameterm{heat} .

If you want to acquire an asset permanently, you can either gain it as a crew upgrade or work on it as a \textbf{long-term project} to set up a permanent acquisition.

\begin{quote}
	Zamira the Whisper is a duelist in the Iruvian style, and would like a fine sword to add to her permanent items. Her player starts a long-term project: “Get My Family Sword Back from the Pawn Shop.” The GM says this is an 8-clock (she can work on it by \gameterm{Consorting}  or \gameterm{Swaying}  the pawn shop owner or maybe rolling her lifestyle level to represent small payments).
\end{quote} 

\section{Long-term project}

When you work on a long-term project (either a brand new one, or an already existing one), describe what your character does to advance the project clock, and roll one of your actions. Mark segments on the clock according to your result: \gameterm{1-3:}  one segment, \gameterm{4/5:}  two segments, \gameterm{6:}  three segments, \gameterm{critical:}  five segments.

A long-term project can cover a wide variety of activities, like doing research into an arcane ritual, investigating a mystery, establishing someone’s trust, courting a new friend or contact, changing your character’s vice, and so on.

Based on the goal of the project, the GM will tell you the clock(s) to create and suggest a method by which you might make progress.

In order to work on a project, you might first have to achieve the means to pursue it---which can be a project in itself. For example, you might want to make friends with a member of the City Council, but you have no connection to them. You could first work on a project to \gameterm{Consort}  in their circles so you have the opportunity to meet one of them. Once that’s accomplished, you could start a new project to form a friendly relationship.

\section{Recover}

When you recover, you seek treatment and heal your harm. You might visit a physicker who can stitch your wounds and soothe your mind with anatomical science or a witch who specializes in healing charms and restorative alchemy. If you don’t have a contact or fellow PC who can provide treatment, you can use the \textbf{acquire asset} activity to gain access to a healer, who can provide service for the whole crew.

Recovery is like a long-term project. Your healer rolls (\gameterm{Tinker}  for a PC with the \gameterm{Physicker}  special ability or the \textbf{quality} level of an NPC) and then you mark a number of segments on your \textbf{healing clock}. \gameterm{1-3:}  one segment, \gameterm{4/5:}  two segments, \gameterm{6:}  three segments, \gameterm{critical:}  five segments.

When you fill your healing clock, reduce each instance of harm on your sheet by one level, then clear the clock. If you have more segments to mark, they “roll over.”

@TODO add table p99

\begin{quote}
	Cross has two injuries: a level 3 “Shattered Right Leg” and level 1 “Battered.” During downtime, he gets treatment from Quellyn, a witch friend of the crew’s Whisper. Quellyn is a competent healer, so the GM says quality 2 makes sense. The player rolls 2d to recover and gets a \gameterm{6} : three segments on the healing clock. He decides to spend 1 \gameterm{coin}  to improve the result to a \gameterm{critical}  to get five segments instead. Four segments fill the clock---all of Cross’s harm is reduced by one level, then he clears the clock and marks one more segment. His level 3 harm “Shattered Right Leg” is reduced to level 2 harm. His level 1 harm “Battered” is reduced to zero and goes away. Cross is left with one injury on his sheet: level 2 “Broken Leg.”
\end{quote} 

You may heal yourself if you have the \gameterm{Physicker}  special ability, but you take 2 stress when you do so. You can also choose to simply tough it out and attempt to heal without any treatment---in this case, take 1 stress and roll 0d.

Note that it’s the recovering character that takes the recovery action. Healing someone else does not cost a downtime activity for the healer.

Whenever you suffer new harm, clear any ticks on your healing clock.

\section{Reduce heat}

Say what your character does to reduce the \gameterm{heat}  level of the crew and make an action roll. Maybe you \gameterm{Consort}  with your friend who’s a officer and she arranges for a few incriminating Watch reports to disappear. Or maybe you \gameterm{Command}  the fear of the local citizens so they’re afraid to snitch.

Reduce \gameterm{heat}  according to the result: \gameterm{1-3:}  one, \gameterm{4/5:}  two, \gameterm{6:}  three, \gameterm{critical:}  five.

\section{Train}

When you spend time in training,\textbf{ mark 1 xp} on the xp track for an \textbf{attribute} or \textbf{playbook }advancement. If you have the appropriate crew Training upgrade unlocked, mark \textbf{+1}\textbf{xp} (2 total). See \textbf{Crew Upgrades}, page 65. \emph{You can train a given xp track only once per downtime.}

\section{Indulge vice}

Visit your \textbf{vice purveyor} to relieve \textbf{stress}. See the next section for details.

\chapter{Vice}

@TODO add table p101

\section{Stress Relief}

Your characters are a special lot. They defy the powers-that-be and dare to prey on those who are considered to be their betters. They push themselves further than ordinary people are willing to go. But this comes at a cost. Their life is one of constant stress. Inevitably, each turns to the seduction of a \textbf{vice} in order to cope.

A character’s vice is their obsession. But with this indulgence comes relief from stress and the ability to once again face the overwhelming challenge of their daring life.

\section{Indulging Your Vice}

When you indulge your vice, you clear some stress from your character’s stress track. Say how your character indulges their vice, including which \textbf{purveyor of vice} they use to satisfy their needs.  This indulgence takes time, so it can only be done when the crew has \textbf{downtime}. Alternately, you may choose to release your character to be “lost in their vice” during a game session, allowing them to indulge off-camera while you play a different PC. A gang member, friend, or contact of the crew might be created as an alternate character to play, thus fleshing out the landscape of PCs.

You roll to find out how much stress relief your character receives. A vice roll is like a resistance roll in reverse---rather than gaining stress levels, you clear stress levels. The effectiveness of your indulgence depends upon your character’s worst attribute rating. It’s their weakest quality (\gameterm{Insight} , \gameterm{Prowess} , or \gameterm{Resolve} ) that is most in thrall to vice.

Make an \textbf{attribute roll} using your character’s lowest attribute rating (if there’s a tie, that’s fine---simply use that rating). \textbf{Clear stress equal to the highest die result.}

\subsection{Overindulgence}

If your vice roll clears more stress levels than you had marked, you overindulge. A vice is not a reliable, controllable habit. It’s a risk---and one that can drive your character to act against their own best interests.

When you overindulge, you make a bad call because of your vice---in acquiring it or while under its influence. To bring the effect of this bad decision into the game, select an overindulgence from the list:

\begin{itemize}
	\item \gameterm{Attract Trouble.}  Select or roll an additional \textbf{entanglement}.
	\item \gameterm{Brag}  about your exploits. \textbf{+2} \gameterm{heat} .
	\item \gameterm{Lost.}  Your character vanishes for a few weeks. Play a different character until this one returns from their bender. When your character returns, they’ve also healed any harm they had.
	\item \gameterm{Tapped.}  Your current purveyor cuts you off. Find a new source for your vice.
\end{itemize}

\subsection{Ignoring your vice}

If you do not or cannot indulge your vice during a downtime phase, you take stress equal to your \gameterm{trauma} . If you don’t have any \gameterm{trauma} , you’re free to ignore your vice. It doesn’t have a hold over you (yet).

\subsection{Roleplaying \& XP}

Along with your character’s heritage and background, their vice tells us what kind of person they are. This obsession impacts their motivations, goals, and behavior. When you ponder what your character might do or say next, you can always consider their vice to help you think of something. As an added benefit, by playing to the nature of your character’s vice, you earn xp at the end of the session.

\chapter{Downtime Activities in Play}

\section{NPC \& faction downtime}

NPCs and factions also do things when the PCs have downtime. The GM \textbf{advances their project clocks} and chooses a downtime maneuver or two for each faction that they’re interested in at the moment. Choose any maneuver that makes sense for that faction to pursue. For example:

\begin{itemize}
	\item Seize a claim or increase hold, make an enemy vulnerable, or reduce the hold of a vulnerable enemy.
	\item Gather information on the PCs (may be opposed by a PC roll) or another subject.
	\item Achieve a short-term goal they’re in position to accomplish.
	\item Acquire a new asset.
	\item Call in a favor from another faction.
	\item Employ political pressure or threats to force someone’s hand.
\end{itemize}

\textbf{GM:} Choose downtime maneuvers and advance clocks for the factions you’re interested in right now. Don’t worry about the rest. Later, when you turn your attention to a faction you’ve ignored for a while, go ahead and give them several downtime phases and project clock ticks to “catch up” to current events.

If you’re not sure how far to progress a faction’s clock, make a fortune roll using their Tier as the base trait, modified up or down depending on the opposition or circumstances. Tick 1 segment for a 1-3 result, 2 segments for a 4/5 result, 3 segments for a 6 result, or 5 segments for a \gameterm{critical}  result.

When factions do things that are known in the criminal underworld, \textbf{tell the players about it} through one of their \textbf{friends} or \textbf{contacts} or \textbf{vice purveyors}. These rumors and bits of gossip can lead to future scores and opportunities for the PCs.

\section{Downtime Activities Summary}

@TODO add tables p102

\chapter{Magnitude}

Supernatural entities and energies have a wide variety of effects and power levels. To help the GM judge these forces consistently, the \textbf{magnitude} scale is provided. See the master table of magnitude at right. Magnitude measures the quality level of a ghost or demon or different aspects of an arcane force: its area, scale, duration, range, and force. Whenever you need to assess an entity or power, use the magnitude scale as a guideline to judge how it compares relative to the examples given on the table.

You can use the magnitude of an entity or power as a dice pool for a \textbf{fortune roll} to see how much effect it has, if it’s not obvious or certain.

\begin{quote}
	A sea demon summons a crushing wave at the canal dock where the PCs are landing their boat. How badly does this damage the vessel and the crew? Obviously it’s gonna be bad for them, but are they merely sinking or are they immediately wrecked and sunk? The GM makes a fortune roll using 6d (the magnitude of the demon). On a \gameterm{1-3} , the wave has only little effect (for a huge wave), causing the craft to take on water and begin sinking. On a \gameterm{4-5} , the wave has reduced effect, fully swamping the boat and throwing some of the characters and their cargo overboard. On a \gameterm{6} , the wave has full effect, immediately sinking the boat and dragging the crew and cargo under. On a \gameterm{critical} , the boat is sunk, and also the crew and cargo are badly harmed by flying debris and the crushing force of the wave.
\end{quote} 

You can add levels of magnitude together to describe a combination of effects, or simply focus on one key feature for the magnitude assessment, ignoring other elements, even if they’re on the magnitude scale. They’re not \emph{always} additive.

\begin{quote}
	In the example above, the demon generated magnitude 6 force and the GM included its area of effect “for free” as part of the power. A huge wave at a dock should affect the boats and the crews there, in their judgment.
\end{quote} 

\begin{quote}
	In a different session, a Whisper wants to accomplish a ritual that will unleash a hurricane across the district. The GM says that this is a very significant effect, so they add two levels of magnitude together: \emph{force 6} and \emph{range 5}. To create such a devastating power, the Whisper will suffer 11 stress! The GM offers a compromise: the ritual will take a few hours to complete, so the stress cost will be reduced to 8, but some people in the affected area may realize what’s happening and flee before the full storm hits.
\end{quote} 

The magnitude table is provided as a tool to help the GM make judgment calls. It’s not meant to be a rigid restriction or mathematical formula to replace those judgment calls. Use the levels as a \emph{guideline} for setting a magnitude number that seem appropriate to you.

This table can also be used as a guide to \textbf{quality level} when a PC \textbf{acquires an asset }or crafts an \textbf{alchemical} or \textbf{gadget}.

@TODO add tables from 105

\chapter{Rituals}

To enact a ritual is to come into contact with these abyssal forces and entreat them to do your will. It is a practice not without considerable risk.

\section{Finding a ritual source}

A PC with the \gameterm{Ritual}  special ability begins with one known ritual, already learned (answer the questions below to create it). To learn a new ritual, a PC must first find a \textbf{source}. A source may be secured as payoff from a score---perhaps you steal a ritual book when your crew robs the Museum of Antiquities. You might also secure a source as the goal of a long-term project---by consorting with cultist friends, studying ancient texts, or some other method you devise.

\section{Learning a ritual}

Once the source of a ritual is found, you may undertake a long-term project to learn the ritual. Most rituals will require an 8-segment progress clock to learn. The player and the GM answer questions about the ritual to define what it will do in play and what is required to perform it (see below). The player records these answers in their notes for future reference.

\subsection{Ritual Questions}
\begin{enumerate}
	\item GM asks: \textbf{“What does the ritual do and how is it weird?”} Player answers.
	\item Player asks: \textbf{“What must I do to perform the ritual, and what is its price?” }GM answers. A ritual takes at least one downtime activity to perform and inflicts \textbf{stress} on the caster according to its \textbf{magnitude}.  If performance of the ritual is dangerous or troublesome in some way, it requires an action roll (usually \gameterm{Attune} ). A ritual may also have additional costs, such as a sacrifice, rare item, the start of a dire progress clock, etc.
	\item GM asks: \textbf{“What new belief or fear does knowledge of this ritual and its attendant occult forces instill in you?”} Player answers.
\end{enumerate}

\subsection{Example Ritual Answers}

\begin{quote}
	Player: “The ritual wards a person so that the ghosts of their victims cannot find them. It’s weird because... as long as the ward is in place, the person sometimes weeps tears of black blood.”
\end{quote} 

\begin{quote}
	GM: “Spend a downtime action to prepare a mixture of tobacco, dream smoke, and crematory ash from a victim---which the target then smokes. You take at least 3 stress when you perform the ritual, which will be its quality for a fortune roll when it’s challenged by a spirit---so you might want to take more stress to make it more potent.”
\end{quote} 

\begin{quote}
	Player: “Gotcha. My new fear is what will happen if the spirits figure out where the ward came from and turn their vengeance on me, instead.”
\end{quote} 

\section{Performing a Ritual}

To perform a ritual, you must have the \gameterm{Ritual}  special ability, then follow the method outlined by the answers to the ritual questions. Most rituals will take \textbf{one downtime activity} to complete, though the GM may call for two (or more) downtime activities for very powerful or far-reaching rituals. Some rituals may be partially performed during downtime and then fully manifested at-will later by completing the last incantation or ritual action. In this case, simply make a note that the ritual has been “primed” and may be unleashed at a later time.

When you perform a ritual, you take an amount of \textbf{stress} as established by the ritual questions, according to the \textbf{magnitude} of the forces brought to bear. The GM uses magnitude as a \emph{guideline} for setting the stress cost---it may be higher or lower at their discretion to better describe the nature of the ritual. Some claims and special abilities also reduce the stress cost for ritual casting (like the Cult’s \textbf{Ancient Obelisk} claim).

\textbf{Rituals take time to cast.} Use the duration examples on the magnitude table to reduce the stress cost based on the time needed, generally no less than an hour.

The GM may also \textbf{tick a progress clock} when you perform a ritual---to advance the agenda of an arcane power or entity, or to show the steady approach of a dark outcome that is a consequence of the ritual’s use.

If a ritual is dangerous or troublesome to perform, make an \textbf{action roll} (usually \gameterm{Attune} ) to see if unpleasant consequences manifest. If a ritual has an uncertain effect then a \textbf{fortune roll} should be made to see how well it manifests. Because a ritual is a downtime activity, you may spend \gameterm{coin}  1-for-1 to increase the result level of your fortune roll (this represents the expenditure of expensive or rare ritual materials). If a ritual is both dangerous and uncertain, then both rolls may be called for.

Each performance of a ritual is a unique event, and may not always work the same way each time. The GM or players may call for a round of questions to establish a ritual anew. Rituals are a way to bring in a wide variety of arcane effects into the game. Use with caution! If you ever go overboard, address the questions again to establish new weirdness and costs if things have gotten out of hand. The abyssal forces are not playthings and cannot be considered a reliable or safe source of power.

\chapter{Crafting}

During downtime, a PC can \gameterm{Tinker}  with special materials and tools to produce strange\textbf{ alchemicals}, \textbf{build} (or \textbf{modify) items},\textbf{ }create \textbf{spark-craft} gadgets, or enchant \textbf{arcane} implements or weapons. The system for each method is similar, with different details depending on the nature of the project.

\section{Inventing}

To invent a formula for a new alchemical concoction or the plan for a new item of your design, you need to \gameterm{Study}  it as a \textbf{long-term project}. Most new formulas or designs will require an 8-segment progress clock to invent and learn. The player and the GM answer questions about the invention to define what it will do in play and what is required to create it (see below). The player records these answers in their notes for future reference.

\subsection{Creation Questions}

\begin{enumerate}
	\item GM asks: \textbf{“What type of creation is it and what does it do?”} Player answers. A creation might be mundane, alchemical, arcane, or spark-craft. If a PC has an appropriate special ability (\gameterm{Alchemist}, \gameterm{Artificer}, \gameterm{Strange Methods} ), they get bonuses when inventing and crafting certain creation types.
	\item Player asks: \textbf{“What’s the minimum quality level of this item?”} GM answers, with the \textbf{magnitude} of the effects the item produces as a guideline.
	\item GM asks: \textbf{“What rare, strange, or adverse aspect of this formula or design has kept it in obscurity, out of common usage?}” Player answers.
	\item Player asks: \textbf{“What drawbacks does this item have, if any?”} GM answers by choosing one or more from the drawbacks list, or by saying there are none.
\end{enumerate}

A PC with the \gameterm{Alchemist} , \gameterm{Artificer} , or \gameterm{Strange Methods}  special abilities invents and learns one special formula when they take the ability (they don’t have to take time to learn it).

Once you’ve invented a formula or design, you can \textbf{craft} it by using a downtime activity (see \textbf{Crafting}, below). No one else can craft this invention unless they learn your design as a long-term project. If you acquire a formula or design invented by another tinkerer, you may learn to craft it by completing a long-term project.

Common alchemicals (see \textbf{Sample Creations} on page 226) and ordinary items don’t require special formulas or designs to learn. Anyone may attempt to craft them by using commonly available instructions.

\section{Crafting}

@TODO add table p109

To craft something, spend one \textbf{downtime activity} to make a \gameterm{Tinker}  roll to determine the \textbf{quality level} of the item you produce. The base quality level is equal to your crew’s Tier, modified by the result of the roll (see the results on the next page).

The results are based on your crew’s Tier because it indicates the overall quality of the workspace and materials you have access to. \emph{If you do the work with the \textbf{Workshop} upgrade for your crew, your effective Tier level is one higher for this roll.}

The GM sets a \textbf{minimum quality} level that must be achieved to craft the item, based on the \textbf{magnitude} of the effect(s) it produces. The GM uses magnitude as a \emph{guideline} for setting the quality level---it may be higher or lower at their discretion to better describe the nature of the project. An item may be crafted at \emph{higher} quality if the player wishes to attempt it.

You may spend \gameterm{coin}  1-for-1 to increase the final quality level result of your roll (this can raise quality level beyond Tier +2).

\subsection{Modifying an item}

Adding a feature or additional function to an item is simpler than creating something new. You don’t need to invent a special formula or plan. Make a crafting roll to modify an item (the baseline quality of an item that you modify is equal to your crew’s Tier, as usual).

\begin{itemize}
	\item A simple, useful modification requires Tier +1 quality. \emph{A rifle that breaks down into two sections to be more easily concealed.}
	\item A significant modification requires Tier +2 quality. \emph{Strengthening the barrel and powder load of a gun to fire further.}
	\item An arcane, spark-craft, or alchemical modification requires Tier +3 quality. \emph{A dagger that can harm a demon. An electrified hull on a boat to repel boarders or ghosts. An outfit coated with chemicals to mask you from deathlands predators.}
\end{itemize}

Modified items, like special creations, may have \textbf{drawbacks}.

\subsection{Drawbacks}

A creation or modification may have one or more drawbacks, chosen by the GM.

\begin{itemize}
	\item \gameterm{Complex. } You’ll have to create it in multiple stages; the GM will tell you how many. One downtime activity and crafting roll is needed per stage.
	\item \gameterm{Conspicuous. } This creation doesn’t go unnoticed. Take \textbf{+1} \gameterm{heat}  if it’s used any number of times on an operation.
	\item \gameterm{Consumable.}  This creation has a limited number of uses (all alchemicals must have this drawback, usually one use).
	\item \gameterm{Rare. } This creation requires a rare item or material when it is crafted.
	\item \gameterm{Unreliable. } When you use the item, make a fortune roll (using its \textbf{quality}) to see how well it performs.
	\item \gameterm{Volatile. } The item produces a dangerous or troublesome side-effect for the user, specified by the GM (see examples on the sample creations, next page). A side-effect is a consequence, and may be \textbf{resisted}.
\end{itemize}

\end{document}
